\section{Functions\label{sec:function}}
A function declaration consists (in the simplest case) of a return type, the function name and comma separated list of argument types (zero or more):
%
\cppline{ReturnType function_name(Type1 arg1, Type2 arg2,...);}
%
A declaration (or prototype) introduces this name into the current scope and tells the compiler that you intend to define it. The
implementation of the body of the function makes a declaration to a definition. Both can be combined or separated. You can even provide
the definition in a separate source file, or even link it dynamically at runtime in your program.

functions that do not return a value have the return type \cpp{void} --- a fundamental type representing the empty set of values.

\begin{rem}
  With \cxx{11}\marginpar{[\cxx{11}]} an alternate syntax for the function declaration is introduced, utilizing the \cpp{auto} keyword:
  %
  \cppline{auto function_name(Type1 arg1, Type2 arg2,...) -> ReturnType;}
  %
  It's call a function with trailing return type. Here the return type might directly depend on the arguments (argument types not values).

  With \cxx{14}\marginpar{[\cxx{14}]} one can even omit the trailing type:
  %
  \cppline{auto function_name(Type1 arg1, Type2 arg2,...) { ... };}
  %
  Here, the declaration must be a definition, since the actual return type is deduced from the \cpp{return} expression of the function.
\end{rem}

% -------------------------------------------------------------------------------------------------
\subsection{Argument passing and return values\label{sec:function-arguments}}
There are essentially two ways how to pass values/arguments to functions:
\begin{description}
  \item[pass-by-value] The passed values initialized a new local variable, that is destroyed at the end of the function body:
  \begin{minted}{c++}
  void foo(int i) {
    i = 5; // the local variable i is changed
  }

  int main() {
    int j = 4;
    foo(j);     // j is not changed
    return 0;
  }
  \end{minted}

  \item[pass-by-reference] Arguments are references to the passed variables and build a local alias of that:
  \begin{minted}{c++}
  void foo(int& i) {
    i = 5; // changes the alias i, thus changing the passed variable j
  }
  int main() {
    int j = 4;
    foo(j);     // j is changed by the function foo
    return 0;
  }
  \end{minted}

  \item[pass-by-const-reference] Similar to \emph{pass-by-reference}, but does not allow to change the value of the alias:
  \begin{minted}{c++}
  void foo(int const& i) {
    /* i = 5; */ // a change of the local reference i is not allowed
  }

  // or
  void foo2(const int& i) {...}

  int main() {
    int j = 4;
    foo(j);     // j can not be changed by foo
    return 0;
  }
  \end{minted}
\end{description}

\begin{rem}
  For small types \cpp{T}, \eg \cpp{int}, \cpp{double}, ..., it is recommended to simple pass-by-value. Larger objects should be passed by
  reference instead, to reduce the cost of copies:
  \begin{enumerate}[1)]
  \item If the function intends to change the argument as a side effect, take it \emph{by non-const reference}.
  \item If the function doesn't modify its argument and the argument is of primitive type, take it \emph{by value}. (``Primitive'' basically means
  small data types that are a few bytes long and aren't polymorphic (iterators, function objects, etc...) )
  \item Otherwise take it \emph{by const reference}, except in the following case:\begin{itemize}
        \item If the function would then need to make a copy of the const reference anyway, take it \emph{by value}.
  \end{itemize}
  \end{enumerate}
\end{rem}

\begin{example}
  Assume, you have a type \cpp{A} that implements a \texttt{+=} operator. In order to implement the plus operator, one can use that. Two implementations
  are possible:
  \begin{minted}{c++}
  // instead of
  A operator+(A const& a, A const& b) {
    A tmp = a; // create a copy of a
    return tmp += b;
  }

  // write:
  A operator+(A a, A const& b) {
    return a += b;
  }
  \end{minted}
  The plus operator can in most cases be implemented as \texttt{+=} and this should be the way to go! Don't repeat yourself.
\end{example}

\begin{guideline}{Principle}
  Make all function arguments \cpp{const} by default, except when intending to change its value.
\end{guideline}

\subsubsection{Default values}
Arguments can be given a default value, that is taken if the argument is not passed to the function. Since parameters are passed in order, default values
must be set to arguments starting from the last one.
\begin{example}
  The following example defines a function with 3 arguments, where the last two are optional:
  \begin{minted}{c++}
  void foo(int a, double b = 2.0, char c = 'c') { /* ... */ }

  int main()
  {
    foo(1); // OK
    foo(1, 5.0); // OK
    foo(1, 5.0, 'x'); // OK
  }
  \end{minted}
\end{example}

\begin{rem}
  One can not directly set a default value for the first argument and no default value for the second argument. If a default is set, for argument $n$, all
  following arguments also must have a default value.

  An alternative to these default values, are \emph{optional} values. This is a library extension in \cxx{17}\marginpar{[\cxx{17}]}, where the actual default
  value is set when you use it:
  \begin{minted}{c++}
  #include <optional>
  void foo(std::optional<int> a_, double b, char c)
  {
    int a = a_.value_or(1);
    /* ... */
  }

  int main()
  {
    foo(1, 5.0, 'x'); // OK
    foo(1, 5.0); // ERROR, char c not passed to function
    foo(std::nullopt, 5.0, 'x'); // OK, a = 1
    foo({}, 5.0, 'x'); // OK, same as above
  }
  \end{minted}
\end{rem}


\subsubsection{Return values}
If a function has a return type unequal to \cpp{void}, it needs a \cpp{return} statement. Void-functions can omit this statement, but are allowed to
call an empty return:
\begin{minted}{c++}
int foo() {
  return 42;
}

void bar() {
  return;
}
\end{minted}

It is allowed to have multiple return statements, but there is only one return type that is fixed by the function declaration. Thus all return statements
must return the same type (or at least all return values are implicitly cast to the return type).

\begin{guideline}{Principle}
  Make sure, that a function has a call to \cpp{return} in every execution path.
\end{guideline}

For the return types it is similar to the argument passing. You can return by value, or return by reference (where for the latter case there
are some pitfalls, see below). But, you can also return a value, by providing an additional \emph{output argument}, \eg
%
\cppline{void function_name(Type1 arg1, Type2 arg, ..., OutputType& output, ...) { ... }}
%
where the order of the function arguments does not matter, \ie you could also start with the output argument. The important aspect is that
output arguments must be passed by (non-const) reference. The intend is that you create the output variable outside of the function and pass it by
reference to the function where it is filled with values.

\begin{rem}
  There is no only-output parameter type in c++, you can always read from the output variable. This is a source of errors so that some people prefer
  not to use output arguments, or use pointer arguments exclusively for this purpose, so that access to the value must be done explicitly.
\end{rem}

A common mistake is to return a reference to a local variable of the function:
\begin{minted}{c++}
std::vector<int>& foo(int input) {
  std::vector<int> temp; // local variable
  // ...
  return temp; // PROBLEM!!!
}
\end{minted}
At the end of the function scope, the local variable is destroyed and thus the reference is referring to some destroyed memory. This is also
called a \emph{dangling reference} and results in an error that is very hard to detect. So, don't do that!

If you return by reference, make sure the object you return outlives the lifetime of the function.

Positive examples:
\begin{minted}{c++}
int& foo1(int& input) {
  return input; // input references an existing object living outside of the function
}
int& foo2() {
  static int i = 5; // a static variable is not destroyed at the end of the function scope
  return i;
}
\end{minted}

\begin{rem}
  Using \cpp{auto} return type might lead to unexpected behavior, if combined with references. The type that is returned is the raw type
  of the expression in the return statement, removing all top-level const and reference qualifiers:
  \begin{minted}{c++}
  auto           f0(int i) { return i; }  // -> int
  auto           f1(int& i){ return i; }  // -> int
  auto&          f2(int& i){ return i; }  // -> int&
  \end{minted}

  The new language keyword \cpp{decltype(auto)}\marginpar{[\cxx{14}]} is introduced, to overcome the problem that returning a reference does not return a reference:
  \begin{minted}{c++}
  decltype(auto) f3(int i) { return i; }  // -> int
  decltype(auto) f4(int& i){ return i; }  // -> int&
  \end{minted}
  It is a placeholder type, as \cpp{auto}, but does not remove top-level const and references.
\end{rem}


\subsubsection{Guidelines}
The CppCoreGuidelines provide a guideline how to pass parameters and how to return values, depending on the type of the arguments:

\begin{figure}[ht]
\begin{center}
\includegraphics[width=.8\linewidth]{images/param-passing-normal}
\end{center}
\caption{From \href{https://github.com/isocpp/CppCoreGuidelines/blob/master/CppCoreGuidelines.md}{CppCoreGuidelines} secion F.15}
\end{figure}


\subsubsection{Return value optimization / Copy elision}
Returning a local value created inside the function by value means naively the same as in argument passing: when the returned value is assigned to a
new variable, this is created as copy of that return value.

This is not completely true. Instead of the procedure  1. create / allocate space for the local variable, 2. create the target variable,
3. copy the function return from the local variable to the target variable, 4. destroy the local variable, the compiler is allowed to use the
fact that the local variable is destroyed at the end of the function, so not used any more, and the target variable is created only to hold
the returned function value. Instead of allocating two variables, it allocates only one and works in both cases with this variable.
With \cxx{17}\marginpar{[\cxx{17}]} this behavior is guaranteed by the standard. Before 2017 standard, it was a compiler optimization.

Example 1:
\begin{minted}{c++}
double f() { return 42.0; }
int main()
{
  double x = f();
  // translates to
  double x = 42.0;
}
\end{minted}

Example 2:
\begin{minted}{c++}
double f()
{
  double y = 42;
  return y; // copy is not guaranteed to be elided, but a common optimization
}

int main()
{
  double x = f();
  // may translate to (named return value optimization)
  double x = 42.0;
}
\end{minted}

Some people prefer the naming \emph{deferred temporary materialization}, \ie the materialization of the value 42 happens in the \cpp{main()} when
assigned to the variable \cpp{x}.

This optimization or guarantee is especially important, if you want to return large objects created inside of the function. It is thus
guaranteed that no expensive copy operation must be performed. (not even a move operation). The created objects simply materializes outside of
the function in the target variable.


% -------------------------------------------------------------------------------------------------
\subsection{Signature\label{sec:function-signature}}
The signature of the function is used to select a function to call. For simple functions, the signature is build of
\begin{itemize}
  \item function name,
  \item parameter-type-list,
  \item enclosing namespace (if any), and
  \item trailing requires-clause (if any) [\cxx{20}]
\end{itemize}
where the \emph{parameter-type-list} specifies the number and type of each individual parameter. Also qualifiers like \cpp{const} and \texttt{\&}
are important in the to distinguish types in that list. It is also allowed to have an \emph{ellipsis}, \ie \texttt{...} at the end of the parameter
list, to indicate a \emph{variadic} function.

The \emph{enclosing namespace} defines the named scope in which the function is defined in and the \emph{trailing requires-clause} is related
to concepts that the parameter types and function must fulfill. (we will discuss this only briefly in a later chapter of the lecture).

The signature + return type defines the type of the function.

\begin{rem}
  An empty parameter list is equivalent to a single parameter of type \cpp{void}. (This is typical in some C programs).
\end{rem}


%--------------------------------------------------------------------------------------------------
\subsection{Function overloading\label{sec:function-overloading}}
In C++ it is allowed to have multiple functions with the same name (in the same scope). At least if it is possible to distinguish those function by
their function signatures! The introduction of the same name multiple times is called \emph{overloading}. Function overloading is used to implement
the same algorithm for different argument types with the same function name.

\begin{example}
Let's consider an example from the beginning:
%
\begin{minted}{c++}
  struct Point { double x, y; };

  double distance(point const& a, point const& b)
  {
    double dx = a.x - b.x;
    double dy = a.y - b.y;
    return std::sqrt(dx * dx + dy * dy);
  }
\end{minted}
%
The function \cpp{distance} is defined for exactly one parameter-type-list, \ie two arguments of type \cpp{Point const&}. But the only requirement we need
in order to calculate the distance with this algorithm is that there are two data member \texttt{.x} and \texttt{.y} that are of some number type.

If you want to calculate the distance of some other point type, the algorithm looks pretty much the same, so it would be logical to use the same function
name for that algorithm.
%
\begin{minted}{c++}
  struct DPoint { double x, y; };
  struct FPoint { float x, y; };

  double distance(DPoint const& a, DPoint const& b)
  {
    double dx = a.x - b.x;
    double dy = a.y - b.y;
    return std::sqrt(dx * dx + dy * dy);
  }

  float distance(FPoint const& a, FPoint const& b)
  {
    float dx = a.x - b.x;
    float dy = a.y - b.y;
    return std::sqrt(dx * dx + dy * dy);
  }

  int main()
  {
    FPoint a, b;
    distance(a,b);
  }
\end{minted}
\end{example}

\begin{rem}
  In the example above there even is the whole body of the function very similar, except for the used value types. In the chapter \emph{Generic Programming}
  we will look into this and instead of writing explicit functions, we will write a function template, parameterized with the value type.
\end{rem}

\begin{guideline}{Principle}
  Functions implementing the same algorithm on different types should be named equal.
\end{guideline}

In order to resolve a function call, \ie to find the right function to evaluate, the compiler has to follow some common steps:
\begin{enumerate}
  \item Look for all functions with the given function name (\emph{name lookup}). All visible function (and names introduced by a \cpp{using}
        declaration/directive) are considered. Thereby the function namespace is important. If the arguments are defined in some namespace also this
        is considered a viable scope for lookup (\emph{Argument Dependent Lookup, ADL}).
  \item If there is more than one viable function found in the name lookup, the compiler tries to resolve the function overload to select a
        function to call (\emph{overload resolution}). Therefore it is considered which parameter-type-list matches best the passed arguments.
  \item All non-matching functions are eliminated from the list of viable candidates (\eg wrong number of arguments or non-matching types). All
        remaining functions are ordered by the property which one is a ``better'' match for the arguments. At first position in that list is the exact
        match (if it exists).
  \item Each pair of viable functions $F_1$ and $F_2$ is ranked by the implicit casts of its arguments, to decide the order. Thereby we have the rule:
        $F_1$ is determined to be a better function than $F_2$ if implicit conversions for all arguments of $F_1$ are not worse than the implicit
        conversions for all arguments of $F_2$, and there is at least one argument of $F_1$ whose implicit conversion is better than the corresponding
        implicit conversion for that argument of $F_2$. Therefore we have the three conversion ranks
        \begin{enumerate}
          \item exact match: no conversion required, lvalue-to-rvalue conversion, qualification conversion (\eg add \cpp{const})
          \item Type promotion: integral promotion, floating-point promotion
          \item Type conversion: integral conversion, floating-point conversion, floating-integral conversion
        \end{enumerate}
\end{enumerate}

\begin{example}
  In the following example the function \texttt{foo()} is overloaded 4 times:
  %
  \begin{minted}{c++}
  #include <iostream>
  struct A { double x = 1.0; };

  int foo(A a) { return 0; }               // (0)
  int foo(A& a) { return 1; }              // (1)
  int foo(A const& a) { return 2; }        // (2)
  int foo(A const& a, int b) { return 3; } // (3)

  int main() {
    A a;
    std::cout << foo(a) << "\n";
  }
  \end{minted}
  %
  This overload-set generates a compiler error, since (0) is equivalent to (1) or (2), \ie all three are exact matches. Removing (0) resolves this error.
  Then (1) is called, since \texttt{a} is a mutable object and thus (1) represents an exact match and no const-qualification is needed.
  (3) is ignored because of the wrong number of arguments.
\end{example}

\begin{example}
  The following example illustrates what it means to be a better candidate:
  %
  \begin{minted}{c++}
  void foo(int const&, short); // overload #1
  void foo(int&, int);         // overload #2

  int main() {
    int i;
    int const ic = 0;
    short s = 0;

    foo(i, 1L);  // 1st argument: i -> int& is better than i -> int const&
                 // 2nd argument: 1L -> short and 1L -> int are equivalent
                 // calls foo(int&, int)

    foo(i,'c');  // 1st argument: i -> int& is better than i -> int const&
                 // 2nd argument: 'c' -> int is better than 'c' -> short
                 // calls foo(int&, int)

    foo(i, s);   // 1st argument: i -> int& is better than i -> int const&
                 // 2nd argument: s -> short is better than s -> int
                 // no winner, compilation error

    foo(ic, 'c'); // 1st argument: ic -> int& not allowed, but ic -> int const& exact match
                  // ---> #1 removed from list of viable functions
                  // 2nd argument: 'c' -> short is allowed
                  // calls foo(int const&, short);
  }
  \end{minted}
\end{example}

\begin{rem}
  There are many more rules for the selection of the best matching function. A detailed explanation can be found on
  \begin{itemize}
	  \item \url{https://accu.org/index.php/journals/268}
	  \item \url{http://en.cppreference.com/w/cpp/language/overload_resolution}
  \end{itemize}
\end{rem}


\subsection{Inline Functions\label{sec:inline-function}}
An \emph{inline}-function is a function qualified with the keyword \cpp{inline}. This has the following consequences:
\begin{itemize}
  \item It is allowed to have multiple definitions of the same (inline)-functions in the same translation unit, unless all definitions are identical.
        (otherwise the behavior is undefined)
  \item The definition must be visible in the translation unit (declaration without definition is not allowed)
\end{itemize}
Further reading: \url{http://en.cppreference.com/w/cpp/language/inline}

Do not mix up the keyword \cpp{inline} with the compiler optimization \emph{inline expansion/replacement}. There, the code of the called function is
expanded at the position of the call instead of performing a jump to the separately defined function. For this to work, the compiler must see the
definition of the function. Advantage of function inlining is the higher locality and the lack of a jump (and jump back). But this also may result in
larger output code. The compiler decides automatically which function to inline. The keyword \cpp{inline} has only very little effect on this decision.
