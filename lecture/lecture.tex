\documentclass[10pt,a4paper]{report}
\usepackage[left=1.5cm,right=2cm,top=2cm,bottom=2cm]{geometry}
\usepackage[utf8]{inputenc}
\usepackage[english]{babel}
\usepackage{amsmath}
\usepackage{amsfonts}
\usepackage{amssymb}
\usepackage{graphicx}
\usepackage{listings}
\usepackage{xcolor}
\usepackage{paralist}
\usepackage{amsthm} % pushQED, popQED
\usepackage{hyperref}
\usepackage{qtree}
\usepackage{tabularx}
\usepackage{minted}
\usepackage{lipsum}
\usepackage{todo}
%\usepackage{dingbat}
\usepackage{mdframed}
\usepackage{imakeidx}
\usepackage{algorithmicx}
\usepackage[noend]{algpseudocode}
\usepackage{tcolorbox}

\usepackage[automark]{scrpage2}

\setlength{\parindent}{0pt}
\setlength{\parskip}{5pt}

\definecolor{hellgrau}{rgb}{0.9, 0.9, 0.9}
\definecolor{dunkelgruen}{rgb}{0, 0.5, 0.0}
\definecolor{grau}{rgb}{0.4, 0.4, 0.4}
\definecolor{dunkelblau}{rgb}{0.3, 0.3, 0.5}
\definecolor{darkgreen}{rgb}{0.0, 0.6, 0.0}
\definecolor{darkblue}{rgb}{0.0, 0.0, 0.8}

\hypersetup{
    bookmarks=false,
    unicode=false,
    pdftoolbar=false,
    pdfmenubar=true,
    pdffitwindow=false,
    pdfstartview={FitH},
    pdftitle={Scientific Programming with C++},
    pdfsubject={Lecture notes}
    pdfauthor={Simon Praetorius},
    pdfnewwindow=true,
    colorlinks=true,
    linkcolor=darkblue,
    citecolor=darkgreen,
    filecolor=magenta,
    urlcolor=darkblue
}

\setminted[c++]{mathescape,
		linenos=false,
		numbersep=5pt,
		frame=none,
		framesep=2mm,
		breaklines=true}


\setminted[java]{mathescape,
		linenos=false,
		numbersep=5pt,
		frame=topline,
		framesep=2mm,
		label={Java Code},
		breaklines=true}

\newcommand{\cpp}{\mintinline{c++}}
\newcommand{\cppline}{\mint[frame=none,linenos=false,breaklines=true]{c++}}
\newcommand{\ie}{i.\,e.,\ }
\newcommand{\eg}{e.\,g.,\ }

\newenvironment{zitat}[1]{%
  \pushQED{#1}%
  \begin{quote}\itshape
}{%
  \par\normalfont\nointerlineskip\noindent\hfill--- \textsc{\popQED}%
  \end{quote}%
}

\newenvironment{guideline}[1]{%
  \pushQED{#1}%
  \begin{tcolorbox}[colback=gray!5]\itshape
}{%
  \par\normalfont\nointerlineskip\noindent\hfill--- \textbf{\popQED}%
  \end{tcolorbox}%
}

\newenvironment{standard}[1]{%
  \pushQED{C++-Standard (\href{http://www.open-std.org/jtc1/sc22/wg21/docs/papers/2019/n4835.pdf}{n4835}) #1}%
  \begin{quote}\itshape
}{%
  \par\normalfont\nointerlineskip\noindent\hfill--- \textsc{\popQED}%
  \end{quote}%
}


\newif\ifmanuscript
\manuscripttrue

\newif\ifoldexcercise
\oldexcercisefalse

\newtheorem{lem}{Lemma}
\newtheorem{thm}{Theorem}
\newtheorem{prop}[thm]{Proposition}
\newtheorem{rem}{Remark}
\newtheorem{example}{Example}

\theoremstyle{definition}
\newtheorem{defn}[thm]{Definition}

\newcommand{\cxx}[1]{\texttt{C++{#1}}}
\newcommand{\TODO}[1]{\begin{mdframed}[backgroundcolor=red!20]\Todo{#1}\end{mdframed}}

\makeindex
\newcommand{\Index}[1]{\textbf{#1}\index{#1}}


\author{Dr. Simon Praetorius}
\title{Scientific Programming with C++}

\begin{document}

\maketitle
\begin{abstract}
The focus of this module lies on aspects of software development like programming on high-performance computers,
object-oriented software design, generic (template-based) programming, and the efficient implementation of
numerical algorithms. Additionally experience in analysis, application and extension of software and software
libraries is developed. This module in the winter term especially focuses on software development with the
programming language C++.

Three main learning goals can be formulated:
\begin{enumerate}
  \item You know how to program with modern C++, using generic programming and advanced techniques, like meta
  programming, expression templates, and concepts.
  \item You know how to use programming tools and you can apply these tools to debug, benchmark, and manage your
  code. The list of tools include compilers, build systems, version control, debuggers, and profilers.
  \item You can read, understand, and utilize (scientific) software libraries, like BLAS (Basic Linear Algebra
  Subroutines), LAPACK (Linear Algebra Package), STL (Standard template library), Dune (framework for the
  discretization of partial differential equations), MTL4 (Matrix Template Library), Boost (portable C++ library).
\end{enumerate}

There are exercises every week to practice the C++ programming. During the semester programming projects in groups
are assigned.

These lecture notes are partially based on a bool manuscript by Peter Gottschling \cite{gottschling2016}, The book
``C++ kurz \& gut'' \cite{loudon2013}, and ``Die C++-Programmiersprache'' \cite{stroustrup2000}. It is continuously
developed since the initial lecture in So2014. This is \textit{work in progress} and may contain typos and errors
that need to be corrected in future releases of the notes. Please feel free to submit any correction.
\end{abstract}

\tableofcontents

\chapter{Introduction}
\begin{zitat}{Bjarne Stroustrup (1994) \cite{stroustrup1994design}}
  ``It would be nice if every kind of numeric software could be written in C++ without loss of efficiency, but unless something can be found that achieves this without compromising the C++ type system it may be preferable to rely on Fortran, Assembler or architecture-specific extensions.''
\end{zitat}

Scientific programming is an old discipline in computer science. The first applications on computers were indeed computations.
In the early decades, ALGOL was a relatively popular programming language, competing with FORTRAN. FORTRAN 77 became a standard
in scientific programming because of its efficiency and portability. Other computer languages were developed in computer science
but not frequently used in scientific computing: C, Ada, Java, C++. They were merely used in universities and labs for research
purposes.

C++ was not a reliable computer language in the 90s: code was not portable, object code not efficient and had a large size. This
made C++ unpopular in scientific computing. This changed at the end of the nineties: the compilers produced more efficient code,
and the standard was more and more supported by compilers. Especially the ability to inline small functions and the introduction
of complex numbers in the C99 standard made C++ more attractive to scientific programmers.

Together with the development of compilers, numerical libraries are being developed in C++ that offer great flexibility together
with efficiency. This work is still ongoing and more and more software is being written in C++. Currently, other languages used
for numerics are FORTRAN 77 (even new codes!), Fortran 95, and Matlab. More and more becoming popular is Python. The nice thing
about Python is that it is relatively easy to link C++ functions and classes into Python scripts. Writing such interfaces is not
a subject of this course, though.

The quote of Bjarne Stroustrup, the developer of the C++ programming language, is from the mid 90s. Since then a lot has changed.
\begin{zitat}{Todd L. Veldhuizen (2000) \cite{veldhuizen2000}}
  ``C++ is now ready to compete with the performance of Fortran. Its performance problems have been solved by a combination of
  better optimizing compilers and template techniques. It's possible that C++ will be faster than Fortran for some applications.''
\end{zitat}

The goal of this course is to introduce students to the exciting world of C++ programming for scientific applications. The course
does not offer a deep study of the programming language itself, but rather focuses on those aspects that make C++ suitable for
scientific programming. Language concepts are introduced and applied to numerical programming, together with the STL and Boost.

It is not our goal to explain all C++ features in a well-balanced manner. We rather aim for an application-driven illustration of
features that are valuable for writing well-structured, readable, maintainable, extensible, type-safe, reliable, portable, and last
but not least highly performing software. In order to achieve this goal we not only want to explain a programming language, but
also discuss best practice in coding and project management of scientific software.

\section{Initial example}
While in classical computer science courses the famous \emph{hello world} example is shown at the beginning, in the scientific
programming lecture I switch to a more numerics oriented initial example: matrix-vector multiplication.

Let $A\in\mathbb{R}^{n\times n}$ be a real (dense) matrix of size $n\times n$, with $n > 0$ a positive integer, and $x\in\mathbb{R}^n$
a corresponding real vector of size $n$. We do not assume any special properties of $A$, like symmetry, bandedness, or triangular
shape. As will be explained later in the lecture and in some exercises, we have two classes, \cpp{DenseMatrix} and \cpp{DenseVector}
that can represent our real-valued matrix $A$ and vector $x$:
%
\begin{minted}{c++}
  DenseMatrix A(100, 100);
  // initialize A
  DenseVector x(100);
  // initialize x
\end{minted}

First, we solve the task $y = A\cdot x + y$ (accumulated matrix-vector multiplication) with $y\in\mathbb{R}^n$ initialized to zero,
using the library CBLAS:
%
\begin{minted}{c++}
  int m = A.rows();         // number of rows of the matrix A
  int n = A.cols();         // number of columns of the matrix A
  int lda = std::max(1,m);  // Specifies the leading dimension of array storing the value of A
  int incx = 1;             // Specifies the increment for the elements of x
  int incy = 1;             // Specifies the increment for the elements of y
  double alpha = 1.0, beta = 1.0;

  DenseVector y(m, 0.0);
  cblas_dgemv(CblasRowMajor, CblasNoTrans, m, n, alpha, &A[0][0], lda,
              &x[0], incx, beta, &y[0], incy);
\end{minted}
%
Second, we solve the same task, using some techniques we will develop during this lecture:
%
\begin{minted}{c++}
  DenseVector y(m, 0.0);
  y += A*x;
\end{minted}

The C++ version more clearly states what it computes, has no arguments that you could mix up, and (this might be
surprising) is not necessarily slower than the first version and even can call all variants of the \texttt{clas\_***mv} methods
from CBLAS simply by analyzing the arguments passed to the multiplication operator \texttt{*}.

\begin{rem}
  If you test the initial exercise implementation, you will find that it is indeed slower than the CBLAS version. During the semester
  we will learn some more advanced techniques and ways to come close the performance of an optimized blas. An example of a high-level
  C++ implementation beating the BLAS call, is MTL4 by Peter Gottschling.
\end{rem}

\begin{rem}
  In BLAS there are multiple different version of the \texttt{dgemv} function. The function name encodes different combinations, i.e.
  the first letter \texttt{d} means \cpp{double} precision, the letters \texttt{ge} means \textit{general}, \texttt{m} mean \textit{matrix}
  and \texttt{v} means vector. So, it implements the matrix-vector product of a general matrix in double precision. Other precision
  types are \{\texttt{s}: single precision, \texttt{d}: double precision, \texttt{c}: complex single precision, and \texttt{z}: complex
  double precision\}, for the matrix type, we have \{\textit{general matrix}, \textit{general band matrix}, \textit{hermitian/symmetric matrix},
  \textit{triangular matrix}, $\ldots$\}, and the matrix can be transposed, or conjugate transposed, and can be in different storage format.

  Thus, BLAS provides many combinations of these different properties, but all functions have a different name, and different arguments. It
  is not so easy to get it right and hard to debug. The documentation of just 3 different functions in all its variants is a 24 pages long document.

  On the other hand in C++ you can encode several of these properties in the type of the matrix. This allows to generate for the \textbf{same}
  function call different specialized implementations by the compiler. So, there is no overhead of dispatching by the various matrix properties.
  The MTL4 library implements this and we will also see how to switch an algorithm by inspecting type properties.
\end{rem}




Ziel dieses Kurses soll es sein über die Grundlagen von C++ hinaus zu erfahren, wie man gut strukturierte, lesbare, wartbare, erweiterbare, Typ-sichere, verlässliche, portable, hoch performante Software mit C++ schreibt. Dabei wollen wir uns in diesem Kurs weniger auf Konzepte des Objekt-Orientierten Programmierens (OOP) konzentrieren, als vielmehr auf Konzepte des generischen Programmierens --- später wird genauer erklärt, was man darunter versteht. Eine Einführung in OOP wird es in einem anderen Kurs \textit{Fortgeschrittene Konzepte des Wissenschaftlichen Programmierens - OOP mit Java} bei Prof. W. Walter zu hören geben.

\section{Organisatorisches}
Der Kurs hat drei Lernziele / Schwerpunkte:
\begin{enumerate}[1)]
\item Konzepte der Programmierung mit C++ (80\%)
\item Arbeit mit Programmierwerkzeugen (z.B. Compiler, Buildsysteme, Versionskontrollsysteme, Debugger, Quelltextdokumentation, Testsysteme) (10\%)
\item Arbeit mit komplexen Software-Bibliotheken (z.B. STL (Standard Bibliothek), Dune (Framework zur Diskretisierung von Differentialgleichungen), MTL4 (Matrix Template Library), Boost (portable C++ Bibliothek) ) (10\%)
\end{enumerate}

Für die Übungen werden wöchentlich Aufgaben online gestellt. Diese sind zu finden unter der Adresse

\url{www.math.tu-dresden.de/~spraetor}

in der Kategorie \textit{Vorlesung / Übung zu Scientific Programming - Fortgeschrittene Konzepte - WS2016}. Mehrfach im Semester werden einzelne Übungsaufgaben ausgesucht, die spätestens zum Ende der jeweiligen Übung abgegeben werden müssen. Diese können aber auch schon zu Hause gelöst werden. Aufgaben sollen dazu in ein Online-Repository eingechecked werden. Mehr dazu in den Übungen.

\section{Literatur}
\begin{description}
\item[Online-Referenzen]: \url{http://www.cplusplus.com}, \url{http://en.cppreference.com}
\item[Grundlage der Vorlesung] Discovering Modern C++: An Intensive Course for Scientists, Engineers, and Programmers, Peter Gottschling, 2015
\item[Referenz] Die C++ Programmiersprache (Bjarne Stroustrup)
\item[Anfänger] C++ Primer (Stanley Lippman, Josee Lajoie, Barbara E. Moo), C++ Kurz \& Gut (Kyle Loudon, Rainer Grimm)
\item[Mittleres Level] (More) Effective C++ (Scott Meyers), C++ Templates: The Complete Guide (David Vandevoorde, Nicolai M. Josuttis)
\item[Fortgeschrittene] Modern C++ Design (Andrei Alexandrescu), C++ Template Metaprogramming (David Abrahams, Aleksey Gurovoy)
\end{description}

\section{Geschichte von C++}
\subsection{Frühes C++}
Die Geschichte von C++ geht zurück auf das Jahr \textbf{1979}, in dem Bjarne Stroustrup sich in seiner Doktorarbeit mit der Programmiersprache \textit{Simula} beschäftigte. Diese gilt als eine der ersten Objekt-Orientierten Programmiersprachen (entwickelt in den \textbf{1960er} Jahren von Ole-Johan Dahl und Kristen Nygaard). Er erkannte den Nutzen dieser Programmiertechnik in der Softwareentwicklung, allerdings war die Simula Sprache viel zu langsam für die praktische Nutzung.

Als erste Entwicklung hin zu C++ wurde in den Jahren \textbf{1979--1983} die Sprache C um Klassen erweitert $\rightarrow$ \textit{C with classes}. Ein erster Kompiler (\textit{Cfront}) wurde von einem C-Kompiler abgeleitet. Die Spracherweiterung umfasste damals Klassen, grundlegende Vererbung, Inlining, Standardwerte in Funktionsargumenten und strenge Type-Überprüfung zusätzlich zu den Features der Sprache C.

Als weitere Features hinzukamen (wie z.B. Exceptions) wurde der Kompiler \textit{Cfront} aufgegeben. Im Jahr 1983 bekam die Sprache dann einen neuen Namen: \textit{C++} der sich vom Inkretment Operator in C ableitete --- also eine Weiterentwicklung von C. Neue Features kamen hinzu: virtuelle Funktionen, Funktionsüberladung, Referenzen mit dem \cpp{&} Symbol, das Schlüsselwort \cpp{const} und einzeilige Kommentare mit dem doppelten Schrägstrich.

Eines der ersten Referenzmanuale war das Buch \textit{The C++ programming language} aus dem Jahr \textbf{1985}, später dann im Jahr 1990 das Buch \textit{The Annotated C++ Reference Manual}. Noch war die Sprache nicht standardisiert, so dass diese Bücher quasi als Standard-Nachschlagewerk dienten. Zu dieser Zeit wurde auch der kommerzielle \textit{Borland C++ Compiler} veröffentlich. Im Jahr \textbf{1987} bekam auch die GNU Compiler Collection GCC Umterstützung für C++.

\subsection{Standard C++}
Der erste ISO-Standard wurde \textbf{1998} veröffentlich, auch bekannt unter der Bezeichnung \cxx{98} Standard. Um das Jahr \textbf{1979} wurde auch begonnen eine Standardbibliothek zu entwickeln, die dann mit in den Standard einfloss. Im Jahr \textbf{1999} wurde \textit{Boost} von einigen Standard-Kommitee-Mitgliedern gegründet, um qualitative Kandidatenten für zukünftige Versionen der C++-Standardbibliothek zu entwickeln.

Eine Korrektur des 98er Standards, die einige Probleme beseitigte, die im Laufe der Zeit entdeckt wurden, ist im Jahr \textbf{2003} veröffentlicht worden. Dieser Standard ist auch bekannt als \cxx{03} Standard. Im Jahre \textbf{2005} wurde eine Erweiterung des 2003er Standards in Form eines technischen Reports TR1 veröffentlicht, der einige neue Features beinhaltet. Eigentlich sollten diese Ideen in einem neuen Standard noch im selben Jahrzehnt zusammengefasst werden, was die Arbeitsbezeichnung \cxx{0x} begründet. Allerdings ist erst in der Mitte des Jahres \textbf{2011} der nächste C++ Standard finalisiert worden (\cxx{11}). Noch immer ist dieser nicht überall verbreitet und viele Programme basieren noch auf dem Standard \cxx{03}. Ab der Version 4.8.1. ist dieser Standard vollständig im Compiler GCC umgesetzt, ab der Version 3.3. im Compiler \textit{clang}, ab der Version 15.0 im Intel C++ Compiler. Microsoft Visual Studio unterstützt \cxx{11} in der aktuellen Version 2015 noch nicht vollständig, aber weitestgehend.

Der C++11 Standard brachte viele neue Features, manche wurden aber nicht vollständig im Standard umgesetzt oder zuende gedacht. Deswegen wurde im Jahr \textbf{2014} eine Minor-Revision zum Standard veröffentlicht (\cxx{14}), die als aktueller Standard angesehen werden kann und von vielen Kompilern nahezu vollständig umgesetzt ist. Der Entwurf von \cxx{14} wird auch als \cxx{1y} bezeichnet. Ab der Kompilerversion 6.1. in GCC wird \cxx{14} vollständig umgesetzt, in Clang ab Version 3.4.

Einen detaillierteren Überblick findet man z.B. unter \url{http://en.cppreference.com/w/cpp/language/history} und in den Büchern von Bjarne Stroustrup: \url{http://www.stroustrup.com/hopl2.pdf} (A History of C++: 1979--1991) und \url{http://www.stroustrup.com/hopl-almost-final.pdf} (Evolving a language in and for the real world: C++ 1991--2006)

\subsection{Zukünftige Entwicklungen}
Für das Jahr \textbf{2017} ist der nächste C++ Standard geplant. Der Entwurf läuft dabei unter der Bezeichnung \cxx{1z} was darauf hindeutet, dass er noch dieses Jahrzehnt veröffentlich wird. Es sind auch schon Ideen für den darauf folgenden Standard \textbf{2020} in verschiedenen Technischen Reports formuliert. Die Dokumente finden sich auf der ISO-Standard Seite der Arbeitsgruppe 21: \url{http://www.open-std.org/jtc1/sc22/wg21/}.

% \section{Motivation}
% Wir werden in den Übungen aus mathematischer Sicht relativ einfache algorithmen betrachten. Dazu zählen die Berechnung des Abstands zweier Punkte, der Norm eines Vektors, Addition, Skalierung und Transformation von Vektoren. Nimmt man diese elementaren Operationen aber zusammen kann man viele komplexe Algorithmen effizient umsetzen. Als Beispiel soll uns hier die klassische Moleküldynamik dienen.
%
% Betrachte $N$ Partikel mit den Massen $m_i$, Positionen $\mathbf{r}_i(t)$ und Geschwindigkeiten $\mathbf{v}_i(t)$ zu einem Zeitpunkt $t$. Die Bewegung und Interaktion dieser Partikel könnte man mit der Newtonschen Bewegungsgleichung beschreiben:
% \[m_k \ddot{\mathbf{r}}_k(t) = \mathbf{F}_k(t, \vec{\mathbf{r}}),\]
% mit $\vec{\mathbf{r}}=(\mathbf{r}_1,\mathbf{r}_2,\ldots,\mathbf{r}_N)$ und $\mathbf{F}_k$ die Kraft, die auf das Partikel $k$ wirkt. Die Kraft wird hier durch eine Entwicklung von Gradienten von Wechselwirkungspotentialen dargestellt, d.h. $\mathbf{F}_k = -\nabla_{\mathbf{r}_k} V(\vec{\mathbf{r}})$, mit
% \[V(\vec{\mathbf{r}}) = \sum_{k=1}^N V_1(\mathbf{r}_k) + \sum_{k<l}^N V_2(\mathbf{r}_k, \mathbf{r}_l) + \sum_{k<l<m}^N V_3(\mathbf{r}_k,\mathbf{r}_l,\mathbf{r}_m) + \cdots\]
% In einigen Anwendungsfällen vereinfacht man die Darstellung des Potentials noch weiter und schränkt sich auf paar-Potentiale $V_2$ ein. Außerdem wird dann häufig angenommen, dass die Wechselwirkung symmetrisch ist, und $V$ nur noch vom Abstand der Partikel abhängt, also
% \[V\approx\sum_{k<l}^N V_2(\|\mathbf{r}_k-\mathbf{r}_l\|),\quad \text{mit }V_2(\|\mathbf{r}_k-\mathbf{r}_l\|)=:V_2(r_{ij}).\]
%
% Das bedeutet, dass zur Simulation der Bewegung von Partikel viele Abstände von Punkten / Koordinaten berechnet werden müssen und dies möglichst effizient. Ein weiterer wichtiger Punkt sind Summen über Ausdrücke. Um dies effizient zu machen, wollen wir uns mit dem Aufrollen von Schleifen beschäftigen.
%
% Hat man viele Partikel und nehmen wir an, dass die Kräfte für große Abstände verschwinden, dann ließe sich über eine Parallelisierung der Simulation nachdenken. Für jeden Partikel wird die selbe Gleichung gelöst und eventuell gibt es nur wenig Wechselwirkung und damit in der Parallelisierung wenig Kommunikation. Im letzten Teil der Vorlesung soll es (wenn es die Zeit zulässt) um das Rechnen auf Grafikkarten und Parallelisierung mittels MPI gehen.

\chapter{C++ Basics\label{sec:basics}}
\section{Introductory example\label{sec:introductory-example}}

Let $x,y\in \mathbb{R}^2$ be two elements of the (Euclidean) vector space $\mathbb{R}^2$, interpreted as
two points (coordinate vectors) in the plane. The space may be equipped with a norm induced by the standard
dot product in $\mathbb{R}^2$.

Implement this vector space and calculate for two elements of the space its distance, i.e. the norm of its difference!

So, the structure representing the vector space should hold all possible realizations of vectors in this vector space
and should provide required vector space operations on its realizations (instances).


\begin{minted}[frame=lines,label={Introductory example}]{c++}
  #include <iostream> // for std::cout
  #include <cmath>    // for std::sqrt

  // Structure defining the vector space
  struct Vector
  {
    double x, y;

    Vector operator-(Vector const& that) const
    {
      return {this->x - that.x, this->y - that.y}; // pointer vs. reference
    }
  };

  double dot(Vector const& a, Vector const& b)
  {
    return a.x * b.x + a.y * b.y; // operator precedence!
  }

  double two_norm(Vector const& a) { return std::sqrt(dot(a,a)); }

  /*
    Distance of two points
    */
  double distance(Vector const& a, Vector const& b) { return two_norm(a - b); }

  int main() // or int main(int argc, char** argv)
  {
    Vector a{ 1.0, 2.0 };
    Vector b{ 7.0,-1.5 };
    std::cout << "distance of a and b = " << distance(a,b) << std::endl;
    return 0;
  }
\end{minted}

Some comments about this example:

\begin{itemize}
  \item Main program: The entry point for the program is the function \cpp{main}. It must be available in all executables and returns an
        integer indicating an error code of the program (\cpp{0} means no error).

  \item Typically, there are just two variants of the \cpp{main} function allowed, without arguments or with two arguments containing
        command-line parameters to the program. Thereby, \cpp{int argc} indicates the number of command-line arguments and \cpp{char** argv}
        or \cpp{char* argv[]} a sequence of null-terminated character sequences (strings) representing the actual command-line arguments.
        Especially, the zero-th argument \cpp{argv[0]} corresponds to the name of the executable.

        \textit{Remark:} Some compilers allow more than those two arguments, that may contain environmental variables.
  \item Input- and Output is not part of the C++ core language, but is implemented in libraries, like the C++ standard library. Those libraries
        must be included explicitly.

  \item Include-files are any regular files that can be found by the compiler. Typically, those includes have the file ending \texttt{.h}
        (for header file) and contain specifications of the interface of functions or even implementations of those functions.
        Include in C++ means: the text of the file is copied to the include directive \cpp{#include} into the code.

        \textit{Remark 1:} Header-files of the C++ standard library do not have a file extension. This is, in order to avoid conflicts with the
        C standard library (those files have the extension \texttt{.h}).

        \textit{Remark 2:} There are two variants for the include directive: \cpp{#include <Dateiname>} or \cpp{#include "Dateiname"}. In the
        first variant the include files are search in the compiler include paths and system paths while in variant 2 it is search also in the
        current source directory. This is why the standard library is typically included in angular brackets \cpp{<...>}.

  \item The functions of the standard library are group in the namespace \cpp{std}. Again, this is done in order to avoid conflicts with functions
        from other libraries and your own code. In order to call functions from the standard library, you have to add the prefix
        (name resolution operator) \cpp{::}, e.g. \cpp{std::sqrt}.

  \item Additionally to the main program, we have a class (structure) \cpp{Vector} and (free) functions \cpp{dot}, \cpp{two_norm}, and \cpp{distance}.
        A free function is a function not part of a class. But, there is also a function bound to the class. It has the strange name
        \cpp{operator-} and implement the subtraction operator between two vectors.

  \item The class \cpp{Vector} contains 2 member variables, that are filled with values for concrete realizations (instances) of that class in the
        \cpp{main()} function. The access to these values is by the ``member selection''-operator \cpp{.} (point).

  \item Finally, the output of the result to the screen is by an output-stream object \cpp{std::cout} (part of the standard library). It provides
        a way to assign new output data to the output device, using the ``shift'' operator \cpp{<<}. The expression \cpp{std::endl} thereby indicates
        a line-break (the end-of-line symbol). One could also write the character \cpp{'\n'} directly.

  \item Brackets \cpp{{ }} in C++ inclose a local code block (scope). Variables declared inside a scope can only be accessed from within that scope
        or a sub-scope. The brackets are also used during the initialization of an object.

  \item There is more in this example: Functions arguments by references, the \cpp{this} pointer inside the class, \cpp{const} specifiers,
        comments \cpp{//} or \cpp{/* ... */},...
\end{itemize}

Some questions to think about:
\begin{itemize}
  \item What happens if you add another \cpp{main(...)} function to the code? Is it possible to add both main functions,
        \cpp{main(), main(int, char**)} at the same time?
  \item There is not just text pushed to the output stream, but also values (numbers). How are numbers printed / converted to strings? How to modify
        this behavior?
  \item What happens if we remove all \cpp{const}s in argument lists? Why is there a \cpp{const} behind a function?
  \item Is it possible to remove the reference sign \cpp{&}? What could be the consequence of this?
\end{itemize}



\section{Compiling C++ code\label{sec:compiling}}
Compared to scripting or interpreted languages, like Python, Matlab, or JavaScript, C++ code must be translated into machine-readable, executable instructions. This process of translation is called \Index{Compiling}. More generally, one could understand compiling as a transformation of code
from one (high-level) language to another (low-level) language.

\begin{rem}
  During the compiling of C++ code, you might even print out intermediate states of is transformation process, like preprocessor output, or
  Assembler output. We will look at these intermediate code in the lecture or exercises, to understand better, what the compiler is doing with
  our code.
\end{rem}

\begin{itemize}
\item The process of compiling is performed by a program, called the \Index{compiler}. Typical examples of compilers are \emph{g++},
      \emph{clang}, \emph{Intel ICC}, \emph{MSVC}, and others.
\item The compiler gets as input a \Index{translation unit}, typically a text file containing the c++ code -- the definition of functions and classes.
      A program typically consists of many translation units that are combined)
\item The output of the compiler is a collection of \Index{object files}, one for each translation unit.
\item To generate an executable (or a library) from these object files, the \Index{linker} combines all the objects to a single file.
\end{itemize}

The process of compiling can be split into several stages:
\begin{description}
  \item[pre-processing] (performed by the \Index{preprocessor}) The content of include files is copied to the include directives, macros and
  preprocessor constants are evaluated.
  \item[linguistic analysis] Check of syntax rules.
  \item[\Index{assembling}] Translation of the language constructs into CPU instructions, e.g. in form of assembler code.
  \item[code output] Transformation of internal code (assembler code) into machine-readable binary code. Collection of symbols into
  symbol table with jump references.
\end{description}

On many linux distributions the c++ compiler of the GNU Compiler Collection (GCC) or the clang compiler of LLVM are preinstalled.
Assume the code from the introductory example is stored in a text file \texttt{distance.cc}. This can be compiled into an executable, by
%
\begin{verbatim}
  c++ distance.cpp
\end{verbatim}
%
where \texttt{c++} is an alias (often a symbolic link) to the actual compiler.

\begin{rem}
  The version and name of the compiler can be obtained by \texttt{c++ --version}.
\end{rem}

The result of the compilation is a binary file, named \texttt{a.out}. This is the default executable name, that can be changed by
providing the argument \texttt{-o NAME}:
%
\begin{verbatim}
  g++ distance.cpp -o distance
\end{verbatim}
%
Later in the lecture we get to know different C++ language features available in a specific version of the C++ standard (see history of C++).
The standard can be selected explicitly by the additional argument \texttt{-std=VERSION}, e.g. for \cxx{11}:
%
\begin{verbatim}
  g++ -std=c++11 distance.cpp -o distance
\end{verbatim}
%
where the \texttt{VERSION} follows the naming given in the chapter \emph{History of C++}.

If you have multiple files to compile, e.g. one file provides the implementation of the functions and classes, and the other file just
the \cpp{main()} function, we say that we have multiple translation units. Those can be compiled individually and then linked together:
%
\begin{verbatim}
  g++ -c file1.cpp
  g++ -c file2.cpp
  g++ file1.o file.o -o program
\end{verbatim}
%
The output name of the compiled translation units follows the pattern \texttt{FILE\_BASE\_NAME.o}. The compiler allows to combine the compiler
and linker call in one line, by listing all the files to compiler one after the other:
%
\begin{verbatim}
  g++ file1.cpp file2.cpp -o program
\end{verbatim}

If a source file depends on some include files (in the top of the file you find the lines \cpp{#include <...>} or \cpp{#include "..."}), the
compiler has to search for these \textit{header}-files. It automatically searches in default system paths, but for everything else the compiler
has to be pointed to the location of the include files. This can be done by the additional argument \texttt{-I[path-to-files]}, e.g.
%
\begin{verbatim}
  g++ -I/usr/local/library/include/ file1.cpp file2.cpp -o program
\end{verbatim}
%
and if the program depends not only on include files, but also \Index{Symbols} (compiled implementations) of library functions, a list of additional
library to link the executable with has to be appended. Therefore, two arguments are allowed for the compiler: \texttt{-L[path-to-library]} and
\texttt{-l[libname]}, where \texttt{libname} contains the part of the file name of the library between the prefix \texttt{lib} and the file
extension \texttt{.so} or \texttt{.a}. (This might be different on different operating systems, like MacOS or MS Windows).
%
\begin{verbatim}
  g++ -I/usr/local/library/include/ file1.cpp file2.cpp -o program -L/usr/local/library/lib -llibrary
\end{verbatim}
%

If your project depends on multiple library that itself depend on other libraries it gets more and more complicated to put everything correctly
into the compile command. To simplify this, there are multiple different \Index{build systems} developed that collect and analyze dependencies
and generate compiler commands for you. A classical one is a \Index{Makefile}, that defines various targets that can depend on each other and
some way to construct from these targets a sequence of commands to execute in order to compile (build) the executable. Another example is
\Index{CMake} (more precisely it is a build system generator).

\begin{rem}
As you may have noticed, source files that are compiled by the compiler are typically named with a file extension \texttt{.cc}, \texttt{.cpp},
or \texttt{.cxx}. This differs from the include (header) files with file extension \texttt{.h}, \texttt{.hh}, \texttt{.hpp}, or \texttt{.hxx}.
Here the first file extension comes from C and is just a abbreviation for \textit{header}. Later in the lecture, we will see source (implementation)
files, that are not compiled, but are typically included at the end of the corresponding header file. This is related to template
implementations. Sometimes these files are name \texttt{.tpp}, or \texttt{.txx}, but more ofter just \texttt{.impl.hh}, or \texttt{.inc.hh},
(with any of the header file extensions from above).

While file extensions and naming of files in general is arbitrary, it is recommended to name source and its corresponding header file with the
same base name and matching file extensions, e.g. \texttt{linear\_algebra.hh} and \texttt{linear\_algebra.cc}.
\end{rem}


% ==============================================================================
\section{Basic structure of a C++ program\label{sec:code-structure}}
Each C++ code resulting in an executable, must contain exactly one \cpp{main(...)} function, while both variants
%
\begin{minted}{c++}
  int main();
  int main(int argc, char* argv[]); // or. int main(int argc, char** argv);
\end{minted}
%
are allowed. The arguments \cpp{argc, argv} are filled when running the executable with command-line arguments. Thereby, the argument \cpp{argc}
represents the number of command-line arguments and \cpp{argv} represents and \textit{array} of \textit{strings} representing each individual
command-line argument. The fist entry in this array, \cpp{argv[0]}, contains the name of the executed program.

\paragraph{Splitting in multiple source files}
Code can (and should) be split into multiple translation units representing different components of the program. This splitting means multiple
header and source files, where each source file can be translated into an object file without the knowledge of the other source files.

Typically, in header files the functions and classes are just \Index{declared}, while in the source file those entities are \Index{defined}.

Example 1: A header file contains the \Index{prototyp} (interface description) of a function and a class definiton.
\begin{minted}[frame=lines,label={example.hh}]{c++}
#ifndef EXAMPLE_HH
#define EXAMPLE_HH

// declaration and definition of a class
struct Point
{
  double x, y;

  // declaration of a member function
  Point subtract(Point const& other) const;
};

// declaration of a function
double distance(Point const& a, Point const& b);

// deklaration of a template function
template <class T> void foo();

#include "example.impl.h"
#endif // EXAMPLE_HH
\end{minted}

Example 2: The definition of a template function (included at the end of the header file)

\begin{minted}[frame=lines,label={example.impl.hh}]{c++}
// definition of the funktion foo()
template <class T>
void foo() { /*...*/ }
\end{minted}

Example 3: The source file, includes the header file and defines the functions

\begin{minted}[frame=lines,label={example.cc}]{c++}
#include "example.hh"
#include <cmath>

// definition of a member function
Point Point::subtract(Point const& other) const
{
  return {this->x - other.x, this->y - other.y};
}

// definition of the function distance()
double distance(Point const& a, Point const& b)
{
  Point ab = a.subtract(b);
  return std::sqrt(ab.x * ab.x + ab.y * ab.y);
}

int main(int argc, char** argv)
{
  Point a{ 1.0, 2.0 }, b{ 7.0,-1.5 };
  distance(a,b);
  return 0;
}
\end{minted}


Some remarks to the examples above:
\begin{itemize}
  \item The triplet \cpp{#ifndef NAME}, \cpp{#define NAME} and \cpp{#endif} builds a so called \textbf{include guard}. It prevents the header file
    to be included multiple times in the same translation unit. This is not allowed, since the C++ standard requires a \textbf{one definition rule},
    meaning: No translation unit shall contain more than one definition of any variable, function, class type, enumeration type, or template.

    Another way of enforcing that a file is included only once, is by using the (non-standard) preprocessor directive \cpp{#pragma once} in the
    top of the include file. This directive is supported by all major compilers and can be used without any problems.

  \item If you want to (or have to) provide an implementation of a function or class method in a header file, it must be included together with
    the corresponding declaration. Often, since is done my an include statement at the end of the header file. Or the definition is provided
    together with the declaration.
\end{itemize}




% ==============================================================================
\section{Variable declaration and fundamental types\label{sec:data-type}}
C++ is a statically typed language (in contrast to dynamically typed languages like e.g. PHP), meaning: each identifier and expression in a
C++ program has assigned a type that is already known to the compiler and this type can not be changed.

Example:
\begin{minted}{c++}
float x;          // x is a single precision floating point number
int y = 3+4;      // y is an integer variable with initial value 7
float f(int);     // f is a function with one argument of type int and returning a float number
\end{minted}

In this example, the variable \cpp{y} is initialized with an expression on the right-hand side of the assignment operator \cpp{=}. This
expression \cpp{3+4} also has a type. Since \cpp{3} and \cpp{4} are integer numbers and the result of the addition of two integers is
defined to be also an integer, the expression is of type \cpp{int}.

\begin{zitat}{C++ standard \S 7 (1)}
  An expression is a sequence of operators and operands that specifies a computation. An expression can result in a value and can cause side effects.
\end{zitat}

\begin{rem}
  That the expression \cpp{3+4} has the type \cpp{int} is not as trivial as you might think. In some languages, it might be a type that could be
  larger than \cpp{int} but that can hold the value of the addition of these two integers.

  Later we will see how to extract the type of an expression (\cpp{decltype}) and how to declare a variable directly of the type
  of the expression assigned to the variable (\cpp{auto}).
\end{rem}

\subsection{Literals\label{sec:literal}}
A literal is a tokes directly representing a constant value of a concrete type.

Examples:
\begin{minted}{c++}
  42u       // unsigned integer literal
  108.87e-1 // floating point literal
  true      // boolean literal
  "Hello"   // string literal
\end{minted}

The type of the literal is often determined by a literal suffix (like in \cpp{42u} the \texttt{u}). All integer literals are just a
sequence of digits that has no period or exponent part, while floating point literals must contain a period and/or an exponent part.
Character literals are introduced with a single quote \cpp{'c'}, and string literals with the double quotes \cpp{"Hello"}. There are
some more literals, we will see in the exercise.

\begin{rem}
  Since \cxx{11} one can define own literals of the form \texttt{build-in literal + \_ + Suffix}. This allows, for example, to create
  numbers with units.

Example:
\begin{minted}{c++}
  101000101_b  // binary representation
  63_s         // seconds
  123.45_km    // kilometer
  33_cent      // cent
\end{minted}

where the implemented is responsible for giving those literals a meaning.
\end{rem}

\begin{rem}
  A literal is a \textit{primary expression}. Its type depends on its form (see above). A string literal is an \Index{lvalue}; all other
  literals are \Index{prvalues}.
\end{rem}


\subsection{Declaration -- Definition -- Initialization}
\begin{description}
\item[Declaration] A declaration may introduce one or more names into a translation unit or redeclare names
  introduced by previous declarations.

\item[Definition] A \textit{declaration} that provides the implementation details of that entity, or in case of variables, reserves memory for
  the entity.

  A \textit{declaration} of a class (\cpp{struct}, \cpp{class}, \cpp{enum}, \cpp{union}), function, or method is a definition if the declaration is
  followed by curly braces, containing the implementation body.

  Variable declarations are always \textit{definitions} unless prefixed with the keyword \cpp{extern}.

\item[Initialization] A \textit{definition} with explicit value assignment.
\end{description}

Examples:
\begin{minted}{c++}
class Test;             // declaration of a class
class Test {};          // definition of that class

int func();             // declaration of a function
int func() { return 7;} // definition of that function

extern int i=func();    // definition and initialization of a variable
extern int j;           // declaration of a variable
int k;                  // definition of a variable

Test obj();             // declaration of a funktion (with return type Test)
\end{minted}


\begin{guideline}{One-definition rule}
  No translation unit shall contain more than one definition of any variable, function, class type, enumeration
  type, or template. (C++-Standard \S 6.2 (1))
\end{guideline}

% =============================================================================
\section{Operators\label{sec:operator}}
Schon im Einführungsbeispiel wurde eine Reihe von Operatoren\index{Operator} verwendet. Operatoren führen bestimmte Operationen auf einer Reihe von Operanden aus. Dabei sind mehr lexikalische Elemente Operatoren, als man vielleicht denkt. Im Beispiel: \texttt{+, -, *, ::, ., <<, = , ",", ()}. Operatoren können auf 1 - 3 Operanden operieren.

Operatoren werden nach drei Eigenschaften charakterisiert: Assoziativität, Präzedenz und Überladbarkeit. Als erstes ist zu bemerken, dass Operatoren in C++ nichts weiter als lexikalische Symbole sind, die eine Infix-Notation für eine Funktion erlauben, z.B. könnte \cpp{a + b} die Infix-Notation für die Funktion \cpp{Result plus(a,b);} sein.

\subsection{Assoziativität\label{sec:operator-associativity}}
Operatorassoziativität bezeichnet:

\begin{zitat}{Wikipedia}
    die Eigenschaft eines Operators, dass die Reihenfolge, mit der mehrere Vorkommnisse dieses Operators in einem Ausdruck ausgewertet werden, keinen Einfluss auf das Ergebnis der Auswertung hat, das heißt, dass für ihn das Assoziativgesetz $(a \circ b) \circ c = a \circ (b \circ c)$ gilt.
\end{zitat}

Erst bei \emph{nicht assoziativen Verknüpfungen} hängt das Ergebnis von der Operatorassoziativität ab. Um zu vermeiden, dass Ausdrücke mit nebeneinander stehenden, gleichwertigen Operatoren ohne Klammerung mehrdeutig sind, wird eine Assoziativität per Konvention festgelegt:
\begin{itemize}
\item Ein \textbf{Linksassoziativer Operator} wird von links nach rechts ausgewertet.
\item Ein \textbf{Rechtsassoziativer Operator} wird von rechts nach links ausgewertet.
\end{itemize}

In C++ kann man Operatoren eine neue Bedeutung (Funktion) geben und damit ist die Assoziativität
nicht mehr gewährleistet. D.h. alle Operatoren werden als links-/rechtsassoziativ klassifiziert.

\begin{example}
Binäre (mathematische) Operationen sind \textbf{linksassoziativ}: (\texttt{+, -, *, /, \%, <, >, \&\&, ||})

\cpp{a + b + c} wird zu \cpp{(a + b) + c}
\end{example}

\begin{example}
Zuweisungsoperatoren sind  \textbf{rechtsassoziativ}:  (\texttt{=, +=, <<=, ...}), d.h. der Ausdruck

\cpp{x= y= z} wird in folgender Reihenfolge interpretiert: \cpp{x= (y= z)}
\end{example}

\subsection{Vorrang (Präzedenz)\label{sec:operator-precedence}}
Operatoren besitzen eine Rangfolge (oder Präzedenz) gegenüber anderen Operatoren. Man definiert damit eine Halbordnung, in der die Operatoren eines in Infix-Schreibweise vorliegenden Ausdrucks auszuwerten sind. Operatoren mit niedriger Präzedenz sind enger an die Argumente gebunden (wie wenn man Klammern setzen würde), als Argumente mit höherer Präzedenz.

Dies kann aber durch das Setzen von zusätzlichen (runden) Klammern modifiziert oder verdeutlicht werden. Klassischerweise gilt, wie in der Mathematik, Punktrechnung vor Strichrechnung, aber bei anderen Operatoren ist dies nicht immer so einfach übertragbar.


\begin{example}
Beispielsweise gibt es einen Operator \cpp{^}, der eigentlich ein logisches X-OR beschreibt. In Matlab/Octave wird dieser Operator aber zum Potenzieren verwendet. Leider hat er in C++ eine andere Präzedenz, als der Potenz-Operator in Matlab (und das Potenzieren in der Mathematik), d.h. der Ausdruck
\cppline{a = b^2 + c}
wird NICHT zu
\cppline{a = (b^2) + c}
sondern zu
\cppline{a = b^(2 + c)}
Im Sinne des Potenzierens ist das natürlich nicht intuitiv und führt schnell zu Fehlern (die auch noch schwer zu finden sind). C++ bietet keinen Operator für das Potenzieren von Ausdrücken.
\end{example}


\subsection{Operatoren als Funktionen}
In C++ ist (fast) jeder Operator als eine Funktion darstellbar. Sei \texttt{\#} das Symbol des Operators, z.B. \texttt{\#$~\in$\{+,*,(),+=,<\}}, dann gibt es eine oder beide der folgenden Implementierungen (oder für manche Operatoren auch keine):

\begin{minted}[frame=none]{c++}
Result operator #(Arg1 a, Arg2 b, ...) // a # b
Result Arg1::operator #(Arg2 b, ...) // Arg1 a; a # b
\end{minted}
Die Variante 1 implementiert den Operator als freie Funktion, in der Variante 2 als Memberfunktion einer Klasse, wobei das \texttt{Arg1} in diesem Fall die Name der Klasse selbst ist. Ob und wie konkret die zugehörigen Funktionen aussehen, findet sich auch in angehängter Tabelle.

Wenn ein Operator als Funktion geschrieben werden kann, dann kann dieser auch überladen\index{Überladung} werden, also für einen eigenen Datentyp spezifiziert/implementiert werden. (Siehe auch Kapitel Funktionsüberladung)

\subsection{Beispiele}
In der Tabelle im Anhang an dieses Kapitel sind die Operatoren von C++, geordnet nach Präzedenz und mit Informationen zur Assoziativität, aufgelistet.

\subsubsection*{Arithmetische Operatoren}
\begin{tabular}{l|l}
Operator & Aktion \\
\hline
\cpp{-} & Subtraktion (unäres Minus) \\
\cpp{+} & Addition (unäres Plus) \\
\cpp{*} & Multiplikation \\
\cpp{/} & Division \\
\cpp{%} & Modulo = Rest bei Division, Für Integer: wenn \cpp{r = a % b}, dann existiert \cpp{c}, so dass \cpp{a=b*c + r}  \\
\cpp{--} & Dekrement (Pre- und Postfix), d.h. \cpp{--a} ist äquivalent zu \cpp{a = a - 1}\\
\cpp{++} & Inkrement (Pre- und Postfix), d.h. \cpp{++a} ist äquivalent zu \cpp{a = a + 1} \\
\end{tabular}
\begin{minted}{c++}
int operator++(int& a, int) { // a++
  int r = a;
  a += 1;
  return r;
}
int& operator++(int& a) { // ++a
  a += 1
  return a;
}
\end{minted}

\subsubsection*{Boolsche Operatoren}
Logische Operatoren und Vergleichsoperatoren

\begin{tabular}{l|l}
Operator & Aktion \\
\hline
\cpp{>} & größer als \\
\cpp{>=} & größer oder gleich \\
\cpp{<} & kleiner als \\
\cpp{<=} & kleiner oder gleich \\
\cpp{==} & gleich \\
\cpp{!=} & ungleich \\
\cpp{&&} & AND \\
\cpp{||} & OR \\
\cpp{!} & NOT \\
\end{tabular}

Ergebnis einer boolschen Operation ist ein boolscher Wert, z.B.
\cppline{bool out_of_bound = x < min_x || x > max_x}

\subsubsection*{Bitweise Operatoren}
Ändern oder Testen von Bits von Integern

\begin{tabular}{l|l}
Operator & Aktion \\
\hline
\cpp{&} & AND \\
\cpp{|} & OR \\
\cpp{^} & exklusives OR \\
\cpp{~} & 1-Komplement \\
\cpp{>>} & Rechts-Shift \\
\cpp{<<} & Links-Shift \\
\end{tabular}
\begin{rem}
Die logischen Operatoren \cpp{<<} und \cpp{>>} werden in C++ häufig zweckentfremdet. Man verwendet sie, um etwas in einen Stream hineinzuschieben oder herauszuholen. Streams sind eine Abstraktion von Geräten, die Ausgabe bzw. Eingabe als Operation erlauben. Der Operator \cpp{<<} wird bei Ausgabestreams auch als Einfügeoperator (insertion operator) bezeichnet, der Operator \cpp{>>} bei Eingabestreams als extraction-operator.
\end{rem}

\begin{task}
Obwohl Rechner oft effiziente Befehle zur Ausführung von arithmetischen und logischen Operationen eingebaut haben, können alle diese Operationen auch durch Kombinationen von bitweisen Operatoren und Nullvergleichen durchgeführt werden. Folgender Pseudocode zeigt beispielsweise, wie zwei beliebige Ganzzahlen a und b nur mithilfe von Verschiebungen und Additionen multipliziert werden können:
\begin{minted}{pascal}
c := 0
solange b <> 0
    falls (b und 1) <> 0
        c := c + a
    schiebe a um 1 nach links
    schiebe b um 1 nach rechts
return c
\end{minted}
Der Code führt eine schriftliche Multiplikation im Binärsystem aus, allerdings in der unüblichen Reihenfolge von hinten nach vorne (beginnend mit der letzten Ziffer von b).\footnote{Siehe \url{https://de.wikipedia.org/wiki/Bitweiser_Operator}}

Implementiere diesen Algorithmus in C++!
\end{task}

\subsubsection*{Zuweisungs-Operatoren}
Compound-Zuweisungs-Operatoren wenden einen Operator auf den Operanden links- und rechts vom Zuweisungsoperator an und speichern das Ergebnis im linken Operanden, d.h.
Operatoren \cpp{+=, -=, *=, /=, %= >>=, <<=, &=, ^=, |=}, mit
\cppline{a += b  // entspricht  a = a + b}
Eine einfache Zuweisung muss eventuell ein Objekt kopieren, also erzeuge \cpp{c = a + b} und weise dies dann \cpp{a} zu: \cpp{a = c}. Mit den kombinierten Zuweisungsoperatoren lässt sich dieser zusätzliche Kopier-Aufwand umgehen, denn
\begin{minted}{c++}
struct A { double value; };

A const operator+(A const& a, A const& b) {
  A c(a); // create local copy of a
  c += b;
  return c;
}
\end{minted}
aber
\begin{minted}{c++}
A& operator+=(A& a, A const& b) {
  a.value += b.value;
  return a;
}
\end{minted}

\begin{rem}
Binäre Operatoren kopieren in der Regel das erste Argument und wenden auf die Kopie die Operation mit dem zweite Argument an. Dies kann so implementiert werden, dass man das Kopieren gar nicht mehr ausschreibt, sondern beim Aufrufen der Funktion automatisch das erste Argument als Kopie übergeben wird:
\begin{minted}{c++}
A const operator+(A a, A const& b) {
  return a += b;
}
\end{minted}
Beachte, dass beim ersten Argument \cpp{a} keine Referenz mehr steht! Das Ergebnis von \cpp{a+=b} liefert das aktualisierte \cpp{a} zurück, dass dann mittels \cpp{return} von der Funktion zurückgegeben wird. Somit kann \cpp{a+b} mittels der Funktion \cpp{a+=b} implementiert werden und sollte auch so geschrieben werden!
\end{rem}

\subsubsection*{Klammer-Zugriffs-Operatoren}
Mit Hilfe der Klammer-Zugriffe \cpp{[]} kann man auf Array Elemente zugreifen und mit \cpp{()} Funktionen aufrufen. Beide Operatoren können überladen werden, um eigene Vektorklassen zu schreiben oder Funktionsobjekte (Funktoren) zu erzeugen.

Mehr zu Funktoren später in der Vorlesung. Auch die Vektorklasse werden wir gesondert betrachten.

\subsection*{Auswertungsreihenfolge und Seiteneffekte}
When an operation has side effects, C++ relies on sequence points rule to decide when side effects (such as increments, combined assignments, etc.) have to take effect. Logical and-then/or-else (\cpp{&&} and \cpp{||}) operators, ternary \cpp{?:} question mark operators, and commas \cpp{,} create sequence points; \cpp{+, -, <<} and so on do not!

\begin{guideline}{Achtung}
When you use an expression with side effects multiple times in the absence of sequence points, the resulting behavior is \textbf{undefined} in C++. Any result is possible, including one that does not make logical sense.
\end{guideline}

Siehe auch: \href{http://en.wikipedia.org/wiki/Sequence_point}{Wikipedia}, \href{http://en.cppreference.com/w/cpp/language/eval_order}{CppReference.com}

Das bedeutet: bei den logischen Verknüpfungen wird zuerst das linke Argument ausgewertet und anhand dessen entschieden, ob das rechte Argument noch ausgewertet werden soll. Z.B. \cpp{A && B}: Zuerst \cpp{A} auswerten. Wenn \cpp{A == true}, dann \cpp{B} auswerten und Ergebnis zurück liefern, wenn \cpp{A == false} wird \cpp{B} nicht ausgewertet. Analog für \cpp{A || B} mit vertauschten Bedingungen. Beim ternären Operator \cpp{A ? B : C} wird zuerst die Bedingung \cpp{A} ausgewertet und anhand dessen Ausgang eines der beiden Ergebniswerte \cpp{B} oder \cpp{C} ausgewertet.

Beim Komma-Operator wird der Reihenfolge nach von links nach rechts ausgewertet.

\begin{rem}
Die Präzedenz hat nichts mit der Auswertungsreihenfolge zu tun, sondern etwas mit dem Vorrang von Operatoren untereinander. Bei einer beliebigen binären (ternären,...) Funktion ist undefiniert, welches der Argumente zuerst interpretiert wird und welches danach.
\end{rem}

\begin{example}
Die beiden folgenden Ausdrücke haben unterschiedliches Verhalten:
\begin{minted}{c++}
int foo(int a, int b) { return a + b; }

int x = 1;
foo(++x, x++); // Variante 1 (Verhalten undefiniert)

x = 1;
int a = ++x, b = x++;
foo(a, b); // Variante 2 (Verhalten festgelegt)
\end{minted}
\end{example}

\subsection*{Übersicht}

\begin{tabular}{c|l|l|l|l}
\textbf{\small Precedence} & \textbf{Operator} & \textbf{Description} & \textbf{Associativity} & \textbf{\small Overload} \\
\hline\hline
1 = highest & \cpp{::}&	Scope resolution                       & Left-to-right & ---\\
\hline
2 & \cpp{++} \cpp{--} &	Suffix/postfix increment and decrement & Left-to-right & $\checkmark$ $\checkmark$ \\
  & \cpp{()}          &	Function call                          & & (K)\\
  & \cpp{[]}          &	Subscript                              & & (K)\\
  & \cpp{. ->}        &	Member access                          & & --- (K)\\
\hline
3 & \cpp{++ --}       & Prefix increment and decrement         &	 Right-to-left & $\checkmark$ $\checkmark$ \\
  & \cpp{+ -}         &	Unary plus and minus                   &   & $\checkmark$ $\checkmark$\\
  & \cpp{! ~}         & Logical NOT and bitwise NOT            &   & $\checkmark$ $\checkmark$\\
  & \cpp{*}           &	Indirection (dereference)              &   & $\checkmark$\\
  & \cpp{&}           &	Address-of                             &   & $\checkmark$\\
\hline
4 & \cpp{.* ->*}      & Pointer-to-member                      & Left-to-right & --- $\checkmark$ \\
\hline
5 & \cpp{* / %}      &	Multiplication, division, and remainder& & $\checkmark$ $\checkmark$ $\checkmark$ \\
\hline
6 & \cpp{+ -}         & Addition and subtraction               & & $\checkmark$ $\checkmark$\\
\hline
7 & \cpp{<< >>}       & Bitwise left shift and right shift     & & $\checkmark$ $\checkmark$\\
\hline
8 & \cpp{< <=}        & For relational operators $<$ and $\leq$ respectively & & $\checkmark$ $\checkmark$\\
  & \cpp{> >=}        &	For relational operators $>$ and $\geq$ respectively & & $\checkmark$ $\checkmark$\\
\hline
9 & \cpp{== !=}       & For relational operators $=$ and $\neq$ respectively & & $\checkmark$ $\checkmark$\\
\hline
10 & \cpp{&}          &	Bitwise AND                            & & $\checkmark$\\
\hline
11 & \cpp{^}          &	Bitwise XOR (exclusive or)             & & $\checkmark$\\
\hline
12 & \cpp{|}          &	Bitwise OR (inclusive or)              & & $\checkmark$\\
\hline
13 & \cpp{&&}         &	Logical AND                            & & $\checkmark$\\
\hline
14 & \cpp{||}         &	Logical OR                             & & $\checkmark$\\
\hline
15 & \cpp{?:}         &	Ternary conditional                    & Right-to-left & ---\\
   & \cpp{=}          &	Direct assignment                      & & (K)\\
   & \cpp{#=}         & Compound assignment operators [note 1] & & $\checkmark$\\
\hline
16 = lowest & \cpp{,}          &	Comma                                  & Left-to-right & $\checkmark$
\end{tabular}

[note 1]: \#$~\in\{$\cpp{+, -, *, /, %, <<, >>, &, ^, |}$\}$

In der Spalte \textit{Overload} bedeutet (K), dass der Operator nur als Memberfunktion einer Klasse überladen werden darf. $\checkmark$ hingegen erlaubt die Überladung als Memberfunktion und freie Funktion.

Eine Übersicht über alle Operatoren, mit Assoziativität, Funktionsdeklaration und Rangfolge, findet sich z.B. auf Wikipedia, unter \url{http://en.wikipedia.org/w/index.php?title=C++\_operators}

\section{Functions\label{sec:function}}
A function declaration consists (in the simplest case) of a return type, the function name and comma separated list of argument types (zero or more):
%
\cppline{ReturnType function_name(Type1 arg1, Type2 arg2,...);}
%
A declaration (or prototype) introduces this name into the current scope and tells the compiler that you intend to define it. The
implementation of the body of the function makes a declaration to a definition. Both can be combined or separated. You can even provide
the definition in a separate source file, or even link it dynamically at runtime in your program.

functions that do not return a value have the return type \cpp{void} --- a fundamental type representing the empty set of values.

\begin{rem}
  With \cxx{11}\marginpar{[\cxx{11}]} an alternate syntax for the function declaration is introduced, utilizing the \cpp{auto} keyword:
  %
  \cppline{auto function_name(Type1 arg1, Type2 arg2,...) -> ReturnType;}
  %
  It's call a function with trailing return type. Here the return type might directly depend on the arguments (argument types not values).

  With \cxx{14}\marginpar{[\cxx{14}]} one can even omit the trailing type:
  %
  \cppline{auto function_name(Type1 arg1, Type2 arg2,...) { ... };}
  %
  Here, the declaration must be a definition, since the actual return type is deduced from the \cpp{return} expression of the function.
\end{rem}

% -------------------------------------------------------------------------------------------------
\subsection{Argument passing and return values\label{sec:function-arguments}}
There are essentially two ways how to pass values/arguments to functions:
\begin{description}
  \item[pass-by-value] The passed values initialized a new local variable, that is destroyed at the end of the function body:
  \begin{minted}{c++}
  void foo(int i) {
    i = 5; // the local variable i is changed
  }

  int main() {
    int j = 4;
    foo(j);     // j is not changed
    return 0;
  }
  \end{minted}

  \item[pass-by-reference] Arguments are references to the passed variables and build a local alias of that:
  \begin{minted}{c++}
  void foo(int& i) {
    i = 5; // changes the alias i, thus changing the passed variable j
  }
  int main() {
    int j = 4;
    foo(j);     // j is changed by the function foo
    return 0;
  }
  \end{minted}

  \item[pass-by-const-reference] Similar to \emph{pass-by-reference}, but does not allow to change the value of the alias:
  \begin{minted}{c++}
  void foo(int const& i) {
    /* i = 5; */ // a change of the local reference i is not allowed
  }

  // or
  void foo2(const int& i) {...}

  int main() {
    int j = 4;
    foo(j);     // j can not be changed by foo
    return 0;
  }
  \end{minted}
\end{description}

\begin{rem}
  For small types \cpp{T}, e.g. \cpp{int}, \cpp{double}, ..., it is recommended to simple pass-by-value. Larger objects should be passed by
  reference instead, to reduce the cost of copies:
  \begin{enumerate}[1)]
  \item If the function intends to change the argument as a side effect, take it \emph{by non-const reference}.
  \item If the function doesn't modify its argument and the argument is of primitive type, take it \emph{by value}. (``Primitive'' basically means
  small data types that are a few bytes long and aren't polymorphic (iterators, function objects, etc...) )
  \item Otherwise take it \emph{by const reference}, except in the following case:\begin{itemize}
        \item If the function would then need to make a copy of the const reference anyway, take it \emph{by value}.
  \end{itemize}
  \end{enumerate}
\end{rem}

\begin{example}
  Assume, you have a type \cpp{A} that implements a \texttt{+=} operator. In order to implement the plus operator, one can use that. Two implementations
  are possible:
  \begin{minted}{c++}
  // instead of
  A const operator+(A const& a, A const& b) {
    A tmp = a; // create a copy of a
    return tmp += b;
  }

  // write:
  A const operator+(A a, A const& b) {
    return a += b;
  }
  \end{minted}
  The plus operator can in most cases be implemented as \texttt{+=} and this should be the way to go! Don't repeat yourself.
\end{example}

\begin{guideline}{Principle}
  Make all function arguments \cpp{const} by default, except when intending to change its value.
\end{guideline}

\subsubsection{Default values}
Arguments can be given a default value, that is taken if the argument is not passed to the function. Since parameters are passed in order, default values
must be set to arguments starting from the last one.
\begin{example}
  The following example defines a function with 3 arguments, where the last two are optional:
  \begin{minted}{c++}
  void foo(int a, double b = 2.0, char c = 'c') { /* ... */ }

  int main()
  {
    foo(1); // OK
    foo(1, 5.0); // OK
    foo(1, 5.0, 'x'); // OK
  }
  \end{minted}
\end{example}

\begin{rem}
  One can not directly set a default value for the first argument and no default value for the second argument. If a default is set, for argument $n$, all
  following arguments also must have a default value.

  An alternative to these default values, are \emph{optional} values. This is a library extension in \cxx{17}\marginpar{[\cxx{17}]}, where the actual default
  value is set when you use it:
  \begin{minted}{c++}
  #include <optional>
  void foo(std::optional<int> a_, double b, char c)
  {
    int a = a_.value_or(1);
    /* ... */
  }

  int main()
  {
    foo(1, 5.0, 'x'); // OK
    foo(1, 5.0); // ERROR, char c not passed to function
    foo(std::nullopt, 5.0, 'x'); // OK, a = 1
    foo({}, 5.0, 'x'); // OK, same as above
  }
  \end{minted}
\end{rem}


\subsubsection{Return values}
If a function has a return type unequal to \cpp{void}, it needs a \cpp{return} statement. Void-functions can omit this statement, but are allowed to
call an empty return:
\begin{minted}{c++}
int foo() {
  return 42;
}

void bar() {
  return;
}
\end{minted}

It is allowed to have multiple return statements, but there is only one return type that is fixed by the function declaration. Thus all return statements
must return the same type (or at least all return values are implicitly cast to the return type).

\begin{guideline}{Principle}
  Make sure, that a function has a call to \cpp{return} in every execution path.
\end{guideline}

For the return types it is similar to the argument passing. You can return by value, or return by reference (where for the latter case there
are some pitfalls, see below). But, you can also return a value, by providing an additional output argument, e.g.
%
\cppline{void function_name(Type1 arg1, Type2 arg, ..., OutputType& output, ...) { ... }}
%
where the order of the function arguments does not matter, i.e. you could also start with the output argument. The important aspect is, that
output arguments must be passed by (non-const) reference. The intend is that you create the output variable outside of the function, pass it by
reference to the function where it is filled with values.

\begin{rem}
  There is no only-output parameter type in c++, you can always read from the output variable. This is a source of errors so that some people prefer
  not to use output arguments, or use pointer arguments exclusively for this purpose, so that access to the value must be done explicitly.
\end{rem}

A common mistake is to return a reference to a local variable of the function:
\begin{minted}{c++}
std::vector<int>& foo(int input) {
  std::vector<int> temp; // local variable
  // ...
  return temp; // PROBLEM!!!
}
\end{minted}
At the end of the function scope, the local variable is destroyed and thus the reference is referring to some destroyed memory. This is also
called a \emph{dangling reference} and results in an error that is very hard to detect. So, don't do that!

If you return by reference, make sure the object you return outlives the lifetime of the function.

Positive examples:
\begin{minted}{c++}
int& foo1(int& input) {
  return input; // input references an existing object living outside of the function
}
int& foo2() {
  static int i = 5; // a static variable is not destroyed at the end of the function scope
  return i;
}
\end{minted}

\begin{rem}
  Using \cpp{auto} return type might lead to unexpected behavior, if combined with references. The type that is returned is the raw type
  of the expression in the return statement, removing all top-level const and reference qualifiers:
  \begin{minted}{c++}
  auto           f0(int i) { return i; }  // -> int
  auto           f1(int& i){ return i; }  // -> int
  auto&          f2(int& i){ return i; }  // -> int&
  \end{minted}

  The new language keyword \cpp{decltype(auto)}\marginpar{[\cxx{14}]} is introduced, to overcome the problem that returning a reference does not return a reference:
  \begin{minted}{c++}
  decltype(auto) f3(int i) { return i; }  // -> int
  decltype(auto) f4(int& i){ return i; }  // -> int&
  \end{minted}
  It is a placeholder type, as \cpp{auto}, but does not remove top-level const and references.
\end{rem}


\subsubsection{Guidelines}
The CppCoreGuidelines provide a guideline how to pass parameters and how to return values, depending on the type of the arguments:

\begin{figure}[ht]
\begin{center}
\includegraphics[width=.8\linewidth]{images/param-passing-normal}
\end{center}
\caption{From \href{https://github.com/isocpp/CppCoreGuidelines/blob/master/CppCoreGuidelines.md}{CppCoreGuidelines} secion F.15}
\end{figure}


\subsubsection{Return value optimization / Copy elision}
Returning a local value created inside the function by value means naively the same as in argument passing: when the returned value is assigned to a
new variable, this is created as copy of that return value.

This is not completely true. Instead of the procedure  1. create / allocate space for the local variable, 2. create the target variable,
3. copy the function return from the local variable to the target variable, 4. destroy the local variable, the compiler is allowed to use the
fact that the local variable is destroyed at the end of the function, so not used any more, and the target variable is created only to hold
the returned function value. Instead of allocating two variables, it allocates only one and works in both cases with this variable.
With \cxx{17}\marginpar{[\cxx{17}]} this behavior is guaranteed by the standard. Before 2017 standard, it was a compiler optimization.

Example 1:
\begin{minted}{c++}
double f() { return 42; }
int main()
{
  double x = f();
  // translates to
  double x = 42;
}
\end{minted}

Example 2:
\begin{minted}{c++}
double f()
{
  double y = 42;
  return y;
}

int main()
{
  double x = f();
  // translates to
  double x = 42;
}
\end{minted}

Some people prefer the naming \emph{deferred temporary materialization}, i.e. the materialization of the value 42 happens in the \cpp{main()} when
assigned to the variable \cpp{x}.

This optimization or guarantee is especially important, if you want to return large objects created inside of the function. It is thus
guaranteed that no expensive copy operation must be performed. (not even a move operation). The created objects simply materializes outside of
the function in the target variable.


%--------------------------------------------------------------------------------------------------
\subsection{References\label{sec:references}}
Although used already, the references need a revisit. References can be understood as alias to (existing) objects. Compared to classical
pointer, they do not represent the address of the references object, but the data of the objects directly.

A reference can be declared like a regular variable, using the reference qualifier \texttt{\&}:
\begin{minted}{c++}
TYPE & VARNAME = aliased_bjekt; // (1)
TYPE && VARNAME= aliased_bjekt; // (2) ... since C++11
\end{minted}
Where (1) is called a lvalue-reference \index{Reference!lvalue reference} and (2) a rvalue-reference \index{Reference!rvalue reference}. A reference
must be initialized directly.

\begin{minted}{c++}
int i = 0;
int& r = i; // r references i
r = 1;      // changes the value of i => i == 1
i = 2;      // also r == 2
\end{minted}

References itself are no objects and do not need own memory. That is why there are no arrays of references and not references of references!
\begin{minted}{c++}
int& a[3]; // error
int& &r;   // error
\end{minted}


\subsubsection{lvalues and rvalues / value categories}
See \url{https://en.cppreference.com/w/cpp/language/value_category}:

Each C++ expression (an operator with its operands, a literal, a variable name, etc.) is characterized by two independent properties:
a \emph{type} and a \emph{value category}. Each expression has some non-reference type, and each expression belongs to exactly one of the three
primary value categories: \emph{prvalue}, \emph{xvalue}, and \emph{lvalue}.
\begin{itemize}
  \item a \emph{glvalue} (``generalized'' lvalue) is an expression whose evaluation determines the identity of an object, bit-field, or function;
  \item a \emph{prvalue} (``pure'' rvalue) is an expression whose evaluation either \begin{itemize}
      \item computes the value of the operand of an operator (such prvalue has no result object), or
      \item initializes an object or a bit-field (such prvalue is said to have a result object). All class and array prvalues have a result object
            even if it is discarded. In certain contexts, temporary materialization occurs to create a temporary as the result object;
  \end{itemize}
  \item an \emph{xvalue} (an ``eXpiring'' value) is a \emph{glvalue} that denotes an object or bit-field whose resources can be reused;
  \item an \emph{lvalue} (so-called, historically, because lvalues could appear on the left-hand side of an assignment expression) is a
        \emph{glvalue} that is not an \emph{xvalue};
  \item an \emph{rvalue} (so-called, historically, because rvalues could appear on the right-hand side of an assignment expression) is a
        \emph{prvalue} or an \emph{xvalue}.
\end{itemize}

\begin{figure}[ht]
\begin{center}
\includegraphics[width=.5\textwidth]{images/value_categories}
\end{center}
\end{figure}

This categorization results from the idea that every expression can be characterized by two orthogonal properties:
\begin{enumerate}
\item[i] \textbf{has identity}: Object that has an address, a pointer, the user can determine whether 2 copies are identical
\item[m] \textbf{can be moved from}: We are allowed to leave the source of a ``copy'' in some indetermined, but valid, state.
\item[$\sim$] has not the property.
\end{enumerate}

The combination \texttt{($\sim$i)($\sim$m)} does not really exists and thus, we have three leaf categories, see Figure.

Examples for lvalues, rvalues, and prvalues:
\begin{minted}{c++}
int i = 3;
int j = 4;

i              // the name of a variable is an lvalue
123, bool, 'c' // literals (except for string literals) are prvalues

// a + b, a % b, a & b, a << b, and all other built-in arithmetic expressions are prvalues

int c = i * j; // OK, (p)rvalue on the right-hand side of an assignment
i * j = 42;    // ERROR, rvalue not allowed on the left-hand side of an assignment

// more on lvalues:
i = 43;           // OK, i is an lvalue
int* p = &i;      // OK, i is an lvalue

int& foo();       // a function call, whose return type is lvalue reference is an lvalue
foo() = 42;       // OK, foo() is an lvalue
int* p1 = &foo(); // OK, foo() is an lvalue

// more on rvalues:
int foobar();        // a function call, whose return type is non-reference is an prvalue
int j = 0;
j = foobar();        // OK, foobar() is an rvalue
int* p2 = &foobar(); // ERROR, cannot take the address of an rvalue
j = 42;              // OK, 42 is an rvalue
\end{minted}


\subsubsection{lvalue- and rvalue references}
We distinguish references to lvalues and rvalues as lvalue-references and rvalue-references, denoted by one or two \texttt{\&}, respectively. So,
lvalue-references refer to existing objects with an identity, i.e. lvalues, whereas rvalues typically refer to something that goes out of scope, who's
lifetime ends. Rvalue references can be used to extend the lifetimes of temporary objects:

\begin{tabular}{p{0.35\textwidth}|p{0.65\textwidth}}
\begin{minted}{c++}
int f() { return 42; }
int& g(int& a) { return a; }

// f() is prvalue expr.
int i = f();  // OK
f()   = 7;    // ERROR

// g() is lvalue expr.
int j = g(i); // OK
g(i)  = 3;    // OK => i == 3
\end{minted}
&
\begin{minted}{c++}
int&  k1 = f(); // ERROR
int&& k2 = f(); // OK
k2 += 1;        // OK, rvalue-Reference extends lifetime

int&  l1 = g(i);      // OK
int&& l2 = g(i);      // ERROR, cannot bind lvalue to
                      // rvalue-References.

// const-References
int const& k3 = f();  // OK
k3 += 1;              // ERROR, const-ref. is non-mutable
int const& l3 = g(i); // OK
\end{minted}
\end{tabular}

In the example, the function \cpp{f()} only returns a temporary object (a number), that is not yet assigned to a variable. Rvalue-references can now
be used to refer to this value with a variable before it is finally destroyed. (This only works for prvalues)

\begin{example}
Example from \href{https://stackoverflow.com/questions/3716277/do-rvalue-references-allow-dangling-references}{Stackoverflow}:
\begin{minted}{c++}
T&  lvalue();
T   prvalue();
T&& xvalue();

T&& does_not_compile = lvalue();
T&& well_behaved = prvalue();
T&& problematic = xvalue();
\end{minted}
\end{example}

\begin{rem}
  The ultimate usage of rvalue-references is, that you know that the object refereed to is going out of scope or is deleted afterwards, or is a
  temporary at all, thus one can directly take the ownership of that object, without copying its content. This is called a move operation.
\end{rem}


% -------------------------------------------------------------------------------------------------
\subsection{Signature\label{sec:function-signature}}
The signature of the function is used to select a function to call. For simple functions, the signature is build of
\begin{itemize}
  \item function name,
  \item parameter-type-list,
  \item enclosing namespace (if any), and
  \item trailing requires-clause (if any) [\cxx{20}]
\end{itemize}
where the \emph{parameter-type-list} specifies the number and type of each individual parameter. Also qualifiers like \cpp{const} and \texttt{\&}
are important in the to distinguish types in that list. It is also allowed to have an \emph{ellipsis}, i.e. \texttt{...} at the end of the parameter
list, to indicate a \emph{variadic} function.

The \emph{enclosing namespace} defines the named scope in which the function is defined in and the \emph{trailing requires-clause} is related
to concepts that the parameter types and function must fulfill. (we will discuss this only briefly in a later chapter of the lecture).

The signature + return type defines the type of the function.

\begin{rem}
  An empty parameter list is equivalent to a single parameter of type \cpp{void}. (This is typical in some C programs).
\end{rem}

%--------------------------------------------------------------------------------------------------
\subsection{Function overloading\label{sec:function-overloading}}
In C++ it is allowed to have multiple functions with the same name (in the same scope). At least if it is possible to distinguish those function by
their function signatures! The introduction of the same name multiple times is called \emph{overloading}. Function overloading is used to implement
the same algorithm for different argument types with the same function name.

\begin{example}
Let's consider an example from the beginning:
%
\begin{minted}{c++}
  struct Point { double x, y; };

  double distance(point const& a, point const& b)
  {
    double dx = a.x - b.x;
    double dy = a.y - b.y;
    return std::sqrt(dx * dx + dy * dy);
  }
\end{minted}
%
The function \cpp{distance} is defined for exactly one parameter-type-list, i.e., two arguments of type \cpp{Point const&}. But the only requirement we need
in order to calculate the distance with this algorithm is that there are two data member \texttt{.x} and \texttt{.y} that are of some number type.

If you want to calculate the distance of some other point type, the algorithm looks pretty much the same, so it would be logical to use the same function
name for that algorithm.
%
\begin{minted}{c++}
  struct DPoint { double x, y; };
  struct FPoint { float x, y; };

  double distance(DPoint const& a, DPoint const& b)
  {
    double dx = a.x - b.x;
    double dy = a.y - b.y;
    return std::sqrt(dx * dx + dy * dy);
  }

  float distance(FPoint const& a, FPoint const& b)
  {
    float dx = a.x - b.x;
    float dy = a.y - b.y;
    return std::sqrt(dx * dx + dy * dy);
  }

  int main()
  {
    FPoint a, b;
    distance(a,b);
  }
\end{minted}
\end{example}

\begin{rem}
  In the example above there even is the whole body of the function very similar, except for the used value types. In the chapter \emph{Generic Programming}
  we will look into this an instead of writing explicit functions, we will write a function template parameterized with the value type.
\end{rem}

\begin{guideline}{Principle}
  Functions implementing the same algorithm on different types should be named equal.
\end{guideline}

In order to resolve a function call, i.e. to find the right function to evaluate, the compiler has to follow some common steps:
\begin{enumerate}
  \item Look for all functions with the given function name (\emph{name lookup}). All visible function (and names introduced by a \cpp{using}
        declaration/directive) are considered. Thereby the function namespace is important. If the arguments are defined in some names also this
        is considered a viable scope for lookup (\emph{Argument Dependent Lookup, ADL} -- see below).
  \item If there is more than one viable function found in the name lookup, the compiler tries to resolve the function overload to select a
        function to call (\emph{overload resolution}). Therefore it is considered which parameter-type-list matches best the passed arguments.
  \item All non-matching functions are eliminated from the list of viable candidates (e.g. wrong number of arguments or non-matching types). All
        remaining functions are ordered by the property which one is a ``better'' match for the arguments. At first position in that list is the exact
        match (if it exists).
  \item Each pair of viable functions $F_1$ and $F_2$ is ranked by the implicit casts of of its arguments, to decide the order. Thereby we have the rule:
        $F_1$ is determined to be a better function than $F_2$ if implicit conversions for all arguments of $F_1$ are not worse than the implicit
        conversions for all arguments of $F_2$, and there is at least one argument of $F_1$ whose implicit conversion is better than the corresponding
        implicit conversion for that argument of $F_2$. Therefor we have the three conversion ranks
        \begin{enumerate}
          \item exact match: no conversion required, lvalue-to-rvalue conversion, qualification conversion (e.g. add \cpp{const})
          \item Type promotion: integral promotion, floating-point promotion
          \item Type conversion: integral conversion, floating-point conversion, floating-integral conversion
        \end{enumerate}
\end{enumerate}

\begin{example}
  In the followign example the function \texttt{foo()} is overloaded 4 times:
  %
  \begin{minted}{c++}
  #include <iostream>
  struct A { double x = 1.0; };

  int foo(A a) { return 0; }               // (0)
  int foo(A& a) { return 1; }              // (1)
  int foo(A const& a) { return 2; }        // (2)
  int foo(A const& a, int b) { return 3; } // (3)

  int main() {
    A a;
    std::cout << foo(a) << "\n";
  }
  \end{minted}
  %
  This overload-set generates a compiler error, since (0) is equivalent to (1) or (2). Removing (0) resolves this error. Then (1) is called, since
  \texttt{a} is a mutable object and thus (1) represents and exact match. (3) is ignored because of the wrong number of arguments.
\end{example}

\begin{example}
  The following example illustrates what it means to be a better candidate:
  %
  \begin{minted}{c++}
  void foo(int const&, short); // overload #1
  void foo(int&, int);         // overload #2

  int main() {
    int i;
    int cont ic = 0;
    short s = 0;

    foo(i, 1L);  // 1st argument: i -> int& is better than i -> int const&
                 // 2nd argument: 1L -> short and 1L -> int are equivalent
                 // calls foo(int&, int)

    foo(i,'c');  // 1st argument: i -> int& is better than i -> int const&
                 // 2nd argument: 'c' -> int is better than 'c' -> short
                 // calls foo(int&, int)

    foo(i, s);   // 1st argument: i -> int& is better than i -> int const&
                 // 2nd argument: s -> short is better than s -> int
                 // no winner, compilation error

    foo(ic, 'c'); // 1st argument: ic -> int& not allowed, but ic -> int const& exact match
                  // ---> #1 removed from list of viable functions
                  // 2nd argument: 'c' -> short is allowed
                  // calls foo(int const&, short);
  }
  \end{minted}
\end{example}

\begin{rem}
  There are many more rules for the selection of the best matching function. A detailed explanation can be found on
  \begin{itemize}
	  \item \url{https://accu.org/index.php/journals/268}
	  \item \url{http://en.cppreference.com/w/cpp/language/overload_resolution}
  \end{itemize}
\end{rem}


\subsection{Inline Functions\label{sec:inline-function}}
An \emph{inline}-function is a function qualified with the keyword \cpp{inline}. This has the following consequences:
\begin{itemize}
  \item It is allowed to have multiple definitions of the same (inline)-functions in the same translation unit, unless all definitions are identical.
        (otherwise the behavior is undefined)
  \item The definition must be visible in the translation unit (declaration without definition is not allowed)
\end{itemize}
Further reading: \url{http://en.cppreference.com/w/cpp/language/inline}

Do not mix up the keyword \cpp{inline} with the compiler optimization \emph{inline expansion/replacement}. There, the code of the called function is
expanded at the position of the call instead of performing a jump to the separately defined function. For this to work, the compiler must see the
definition of the function. Advantage of function inlining is the higher locality and the lack of a jump (and jump back). But this also may result in
larger output code. The compiler decides automatically which function to inline. The keyword \cpp{inline} has only very little effect on this decision.


\section{References and Pointers\label{sec:references}}
Although used already, the references need a revisit. References can be understood as alias to (existing) objects. Compared to classical
pointer, they do not represent the address of the references object, but the data of the objects directly.

A reference can be declared like a regular variable, using the reference qualifier \texttt{\&}:
\begin{minted}{c++}
TYPE & var2  = aliased_objekt; // (1)
TYPE && var2 = aliased_objekt; // (2) ... since C++11
\end{minted}
Where (1) is called a lvalue-reference \index{Reference!lvalue reference} and (2) a rvalue-reference \index{Reference!rvalue reference}. A reference
must be initialized directly and since it is not possible to change where reference refers to.

\begin{minted}{c++}
int i = 0;
int& r = i; // r references i
r = 1;      // changes the value of i => i == 1
i = 2;      // also r == 2
\end{minted}

References itself are no objects and do not need own memory. That is why there are no arrays of references and no references of references!
\begin{minted}{c++}
int& a[3]; // error
int& &r;   // error
\end{minted}


% -------------------------------------------------------------------------------------------------
\subsection{lvalues and rvalues / value categories}
See \url{https://en.cppreference.com/w/cpp/language/value_category}:

Each C++ expression (an operator with its operands, a literal, a variable name, etc.) is characterized by two independent properties:
a \emph{type} and a \emph{value category}. Each expression has some non-reference type, and each expression belongs to exactly one of the three
primary value categories: \emph{prvalue}, \emph{xvalue}, and \emph{lvalue}.
\begin{itemize}
  \item a \emph{glvalue} (``generalized'' lvalue) is an expression whose evaluation determines the identity of an object, bit-field, or function;
  \item a \emph{prvalue} (``pure'' rvalue) is an expression whose evaluation either \begin{itemize}
      \item computes the value of the operand of an operator (such prvalue has no result object), or
      \item initializes an object or a bit-field (such prvalue is said to have a result object). All class and array prvalues have a result object
            even if it is discarded. In certain contexts, temporary materialization occurs to create a temporary as the result object;
  \end{itemize}
  \item an \emph{xvalue} (an ``eXpiring'' value) is a \emph{glvalue} that denotes an object or bit-field whose resources can be reused;
  \item an \emph{lvalue} (so-called, historically, because lvalues could appear on the left-hand side of an assignment expression) is a
        \emph{glvalue} that is not an \emph{xvalue};
  \item an \emph{rvalue} (so-called, historically, because rvalues could appear on the right-hand side of an assignment expression) is a
        \emph{prvalue} or an \emph{xvalue}.
\end{itemize}

\begin{figure}[ht]
\begin{center}
\includegraphics[width=.5\textwidth]{images/value_categories}
\end{center}
\end{figure}

This categorization results from the idea that every expression can be characterized by two orthogonal properties:
\begin{enumerate}
\item[i] \textbf{has identity}: Object that has an address, a pointer, the user can determine whether 2 copies are identical
\item[m] \textbf{can be moved from}: We are allowed to leave the source of a ``copy'' in some indetermined, but valid, state.
\item[$\sim$] has not the property.
\end{enumerate}

The combination \texttt{($\sim$i)($\sim$m)} does not really exists and thus, we have three leaf categories, see Figure.

Examples for lvalues, rvalues, and prvalues:
\begin{minted}{c++}
int i = 3;
int j = 4;

i              // the name of a variable is an lvalue
123, bool, 'c' // literals (except for string literals) are prvalues

// a + b, a % b, a & b, a << b, and all other built-in arithmetic expressions are prvalues

int c = i * j; // OK, (p)rvalue on the right-hand side of an assignment
i * j = 42;    // ERROR, rvalue not allowed on the left-hand side of an assignment

// more on lvalues:
i = 43;           // OK, i is an lvalue
int* p = &i;      // OK, i is an lvalue

int& foo();       // a function call, whose return type is lvalue reference is an lvalue
foo() = 42;       // OK, foo() is an lvalue
int* p1 = &foo(); // OK, foo() is an lvalue

// more on rvalues:
int foobar();        // a function call, whose return type is non-reference is an prvalue
int j = 0;
j = foobar();        // OK, foobar() is an rvalue
int* p2 = &foobar(); // ERROR, cannot take the address of an rvalue
j = 42;              // OK, 42 is an rvalue
\end{minted}


% -------------------------------------------------------------------------------------------------
\subsection{lvalue- and rvalue references}
We distinguish references to lvalues and rvalues as lvalue-references and rvalue-references, denoted by one or two \texttt{\&}, respectively. So,
lvalue-references refer to existing objects with an identity, i.e. lvalues, whereas rvalues typically refer to something that goes out of scope, who's
lifetime ends. Rvalue references can be used to extend the lifetimes of temporary objects:

\begin{tabular}{p{0.35\textwidth}|p{0.65\textwidth}}
\begin{minted}{c++}
int f() { return 42; }
int& g(int& a) { return a; }

// f() is prvalue expr.
int i = f();  // OK
f()   = 7;    // ERROR

// g() is lvalue expr.
int j = g(i); // OK
g(i)  = 3;    // OK => i == 3
\end{minted}
&
\begin{minted}{c++}
int&  k1 = f(); // ERROR
int&& k2 = f(); // OK
k2 += 1;        // OK, rvalue-Reference extends lifetime

int&  l1 = g(i);      // OK
int&& l2 = g(i);      // ERROR, cannot bind lvalue to
                      // rvalue-References.

// const-References
int const& k3 = f();  // OK
k3 += 1;              // ERROR, const-ref. is non-mutable
int const& l3 = g(i); // OK
\end{minted}
\end{tabular}

In the example, the function \cpp{f()} only returns a temporary object (a number), that is not yet assigned to a variable. Rvalue-references can now
be used to refer to this value with a variable before it is finally destroyed.

\begin{example}
Example from \href{https://stackoverflow.com/questions/3716277/do-rvalue-references-allow-dangling-references}{Stackoverflow}:
\begin{minted}{c++}
T&  lvalue();
T   prvalue();
T&& xvalue();

T&& does_not_compile = lvalue();
T&& well_behaved = prvalue();
T&& problematic = xvalue();
\end{minted}
\end{example}

\begin{rem}
  The ultimate usage of rvalue-references is, that you know that the object refereed to is going out of scope or is deleted afterwards, or is a
  temporary at all, thus one can directly take the ownership of that object, without copying its content. This is called a move operation.
\end{rem}


% -------------------------------------------------------------------------------------------------
\subsection{Pointers}
Referencing to something in memory could be understood as just storing the address of that memory. An address ist just an integer indicating
the position in memory relative to some initial address. But, we need more than just the adress, we need the type of the data that we are
referring to, to that we can give the memory a meaning. This is called a pointer:
%
\cppline{TYPE * pointer = &OBJECT;}
%
where \texttt{OBJECT} is any \underline{lvalue-reference} type.
We have used the \texttt{\&} symbol before, to qualify/indicate a reference. In the context of pointers, this symbol is an (unary) operator, the
address-of operator, returning a pointer to the object in memory. A pointer type is indicated by the \texttt{*} qualifier.

The \textit{dual} operator to the address-of operator is the de-reference operator, giving a reference to the object, a pointer points to:
%
\cppline{TYPE & reference = *POINTER;}
%
where \texttt{POINTER} must be any object of pointer type. So, the dereferenced pointer is again an lvalue.

Pointers can be combined with \cpp{const} qualifiers, to either indicate that the addressed data is immutable or the the address value is constant.
Which one you mean is determined by the position of the \cpp{const} qualifier, i.e.
%
\begin{minted}{c++}
  int data = 42, data2 = 1234;

  int const* i1 = &data;
  *i1 = 7;                // ERROR: can not modify constant data
  i1 = &data2;            // OK: i1 now points to data2

  int* const i2 = &data;
  *i2 = 7;                // OK: data is not const
  i2 = &data2;            // ERROR: can not change address of const pointer

  int const* const i3 = &data; // completely imutable
\end{minted}

Since one can not have a reference of a reference, one can also not declare a pointer to a reference (something like \cpp{int&*}), but a
pointer is a regular type representing an address, it can be referenced, i.e.
%
\begin{minted}{c++}
  int data = 42, data2 = 1234;
  int* p = &data;
  int*& r = p;  // reference to pointer p

  *r = 7;       // changes value of data;
  r = &data2;   // changes stored address of p
\end{minted}

\begin{guideline}{Guideline}
  Don't use pointers to pass data to functions, only if you have a strong reason to do so. Some style-guides use pointers as output arguments
  in functions to not mix it up with input arguments. This is, because you have to explicitly de-reference the pointer in order to use it.
\end{guideline}

\section{Arrays and Dynamic Memory}
\subsection{Static Arrays}
So far, we had only data-types with 1 entry per variable (except for the library types \texttt{vector}, \texttt{pair} and \texttt{tuple}).
Arrays are compound data-types with a \underline{fixed} number of entries of the same type:
%
\cppline{TYPE variable[ SIZE ];}
%
where \texttt{SIZE} is an \emph{integral constant expression}.

Examples:
\begin{minted}{c++}
  int n[5];
  double f[ 10 ];
  constexpr int len = 32;
  char str[len];
\end{minted}
%
Those arrays can then be accessed by square-brackets, similar to \cpp{std::vector}, \eg
%
\cppline{n[2] = 7;}
%
Arrays can be pre-initialized. In that case, the array size can be omitted, \ie is detected
automatically by the compiler:
\begin{minted}{c++}
  int n1[5] = {1, 2, 3, 4, 5};
  int n2[]  = {10, 9, 8}; // automatically size 3
\end{minted}

Multi-dimensional arrays are written with multiple square brackets:
%
\begin{minted}{c++}
  int A[2][3]; // a 2 x 3 matrix
  double B[5][4][3]; // an 5 x 4 x 3 tensor
\end{minted}
%
and can be understood as arrays of arrays (a matrix is and arrays of rows).

The initialization of multi-dimensional arrays is by lists of initializer-lists, \eg
%
\begin{minted}{c++}
double A[2][3] =
{
  {1.0, 2.0, 3.0},
  {4.0, 5.0, 6.0}
};
\end{minted}

\begin{rem}
  An array can be pre-initialized with less elements than the given size. In that case the remaining entries are default-initialized, \ie
  number types are initialized to zero and class-types using the default-constructor.

  Example:
  \cppline{int foo[5] {1,2,3}; // [1,2,3,0,0]}
  (The assignment operator ``\cpp{=}'' is here replaced by \textit{Universal Initialization}.)
\end{rem}

A multi-dimensional array can be written with a one-dimensional initializer-list. In that case, the initialization happens arrays-wise.
%
\cppline{int foo[][2] {1,2,3,4}; // [1,2; 3,4]}
%
For multi-dimensional arrays only one size entry can be omitted. The shape must be deducible by the number of entries.


\subsubsection{Size of an array}
While the syntax for arrays is simple and clean, it lacks the direct possibility to get the size after the declaration. In C, you typically
use macros to deduce the size from the byte-size of the array, \ie
%
\cppline{#define SIZE(a) (sizeof(a) / sizeof((a)[0])) }
%
Note the usage of additional brackets, here.

Example:
\begin{minted}{c++}
  int vec[5] {1,2,3};
  static_assert(SIZE(vec) == 5, "");

  int mat[][2] {1,2,3,4};
  static_assert(SIZE(mat) == 4, "");
\end{minted}

One can not directly determine the rank of a matrix/tensor and thus number of rows and columns of a matrix without prior knowledge of the shape
is not possible to detect with macros.


\subsubsection{Arrays as function arguments}
Arrays can be (implicitly) converted to a pointer, \ie to a pointer to the first element of the array:
%
\begin{minted}{c++}
int a[3] = {1, 2, 3};
int* p = a;  // pointer to the first element of the array
\end{minted}
%
This allows to pass arrays to functions, accepting pointers as arguments, \ie
%
\begin{minted}{c++}
  void f1(int* p) { /* sizeof(p) != sizeof(a) */ }
  void f2(int p[]) { /* sizeof(p) != sizeof(a) */ }

  int a[3] = {1, 2, 3};
  f1(a); // OK: a is converted to a pointer
  f2(a); // OK: equivalent to f1
\end{minted}
%
But, then we loose any information about the array. It is just a pointer to the first element of that array.

Nevertheless, pointers to array data allow direct element access, like for the regular arrays:
%
\cppline{int a1 = p[1];}

The second form of passing arrays to functions is by array reference
%
\cppline{TYPE (&reference)[SIZE] = ARRAY;}
%
Example:
\begin{minted}{c++}
  void g(int (&b)[3]) { /* sizeof(b) == sizeof(a) */ }

  int a[3] = {1, 2, 3};
  g(a);
\end{minted}
The problem here, we have to give the size in the function argument declaration, since the size of the arguments must be known
at compile-time and are fixed to exactly one size. It is not allowed to pass arrays of different size to that function.

\begin{rem}
\underline{Outlook:} Using \emph{templates}, we can write functions of arbitrary array size and can deduce that size automatically, \ie
\begin{minted}{c++}
  template <class T, int N>
  int array_size(T (&)[N]) { return N; }

  template <class T, int N1, int N2>
  std::pair<int,int> mult_array_size(T (&)[N1][N2]) { return {N1, N2}; }
\end{minted}
\end{rem}


% -------------------------------------------------------------------------------------------------
\subsection{Standard Arrays}
The problem with the size of the arrays, the argument passing to functions (and some more restriction) motivates to introduce a library type
that overcomes those problems. With \cxx{11}\marginpar{[\cxx{11}]} the type \cpp{std::array<T, Size>}\index{Array} was added to the standard
library:
\begin{minted}{c++}
  #include <array>
  // ...
  std::array<int, 3> vec1 = { 1.0, 2.0, 3.0 };
  std::array vec2 = {1, 2, 3}; // with c++17

  std::cout << vec1[1];
\end{minted}

The difference to the other library type \cpp{std::vector} is, that an array must have a fixed size, and the data is allocated to the ``stack''
and not dynamic memory allocated on the ``heap'' as in the vector implementation (see below).

\begin{guideline}{Principle}
  Prefer the library type \cpp{std::array} over classical arrays.
\end{guideline}


% -------------------------------------------------------------------------------------------------
\subsection{Dynamic memory allocation}
We have seen that arrays can be implicitly converted to pointers and pointers can be used similar to arrays. Can we allocate an array directly as
a pointer? Is it possible to have an array of dynamic size? One needs to initialize a pointer with the address of a memory block large enough to hold
the wanted array. This is accomplished by \emph{dynamic memory management}: memory of arbitrary size can be allocated and deallocated at runtime!

In C++ this is done with the operator \cpp{new} and \cpp{new[]} to allocate memory and construct objects in the newly allocated memory, and \cpp{delete} and \cpp{delete[]} to destruct the objects and deallocate memory.

For a single element:
\begin{minted}{c++}
  TYPE * p = new TYPE;  // allocation of one element of type TYPE
  delete p;             // deallocation of one element of type TYPE
\end{minted}
and for more than one element:
\begin{minted}{c++}
  TYPE * p = new TYPE[ SIZE ];
  delete[] p;
\end{minted}

Note, the \texttt{SIZE} parameter to \cpp{new[]} does not need to be a constant expression!

Examples:
\begin{minted}{c++}
  char* s = new char[100];    // dynamic array of 100 elements
  int n = 1024;
  double* v = new double[n];  // dynamic size
  float* f = new float;       // single element

  for (int i = 0; i < n; ++i)
    v[i] = i*i;

  *f = 1.41421356237;         // dereference the pointer to get a reference

  delete[] v;   // new[] => delete[]
  delete[] s;   // new[] => delete[]
  delete f;     // new   => delete
\end{minted}

One problem with dynamic arrays is that there is no way to get the size of that array, and there is no way for the compiler
to perform a boundary check. Access of a memory location outside of the allocated array size may lead to a \emph{segmentation fault} or
may simply overwrite the data stored at the specific memory location $\rightarrow$ hard to find errors.

\begin{guideline}{Rule}
  \cpp{delete} (or \cpp{delete[]}) can be applied to pointers created by the corresponding operator \cpp{new} (or \cpp{new[]}). On other pointers
  it may lead to undefined behavior (and often produces a runtime-error of the form ``invalid pointer'').
\end{guideline}

Another problem with dynamic arrays is that you can not even see whether it is an allocated array, or just an uninitialized pointer variable. So,
here again the general rule applies: initialize variables directly on declaration. If you do not yet have an address to initialize the pointer with,
set it to zero, \ie
%
\cppline{TYPE * pointer = nullptr;}
%
where \cpp{nullptr} is a special object just for this case of initialization of pointers. After \cpp{delete}, reset the pointer to \cpp{nullptr} again.

Example:
\begin{minted}{c++}
  int* p = nullptr;
  p = new int[1234];
  // ...
  delete[] p;
  p = nullptr;

  if (p) {  // test for p != nullptr
    /* work with p */
  }
\end{minted}

Who owns the memory of a pointer, \ie who is responsible for deleting that pointer? It is the one that has created the array, so do not forget to delete!
Otherwise you have a \emph{memory leak}. The aftermath of a pointer related bug, \eg array boundary violation or accessing deleted memory, may show up \underline{much later} than the actual position of the error.

\begin{guideline}{Principle}
  Pointers are \emph{dangerous} and require careful programming. So, prefer library types like \cpp{std::vector} or \cpp{std::array} instead of
  (dynamic/static) arrays.
\end{guideline}


\subsubsection{Pointer arithmetics}
Pointers used as addresses of arrays, \ie the address to the first element of the array, have its one arithmetic. Since addresses are like integers,
they can be incremented and decremented, meaning a movement of the pointer in memory.

Pointers are associated to a specific element type (underlying type of the data). Incrementing and decrementing thus moves the address by multiples of
the size of the underlying data. This means that \cpp{pointer + 1} points to the next element in an array, where \texttt{pointer} points to.

Examples:
\begin{minted}{c++}
int* vec = new int[10]; // vec points to the first element of the array
vec = vec + 1;          // vec now contains the address (vec + sizeof(int)), \ie
                        // vec points to the second element of the array
vec++;                  // vec points to the third element of the array...
delete[] vec;           // ERROR: Address of vec not allocated by a direct call to new[]
\end{minted}

Pointers can be incremented (\texttt{+, ++}), or decremented (\texttt{-, --}), pointers can be compared, comparing the underlying address.
Pointers can be dereferenced, giving a reference to the data it points to. Additionally, pointers have the subscript-operator \cpp{[n]} that is
equivalent to \cpp{*(pointer + n)}.

Requirement for using pointer arithmetic is that the data has a defined size (complete types). The ``empty-set'' type \cpp{void} can be used as a pointer type,
meaning just the address without any reference to the data-type of the stored data. Since \cpp{void} does not have a size, pointers to \cpp{void} can not
be used in pointer arithmetics.


% -------------------------------------------------------------------------------------------------
\subsection{Storage duration}
All objects in a program have one of the following storage durations:
\begin{itemize}
  \item \textbf{automatic} storage duration. The storage for the object is allocated at the beginning of the enclosing code block and deallocated at the end. All local objects have this storage duration, except those declared \cpp{static}, \cpp{extern} or \cpp{thread_local}.

  \item \textbf{dynamic} storage duration. The storage for the object is allocated and deallocated per request by using dynamic memory allocation functions (\cpp{new} and \cpp{delete}).

  \item \textbf{static} storage duration. The storage for the object is allocated when the program begins and deallocated when the program ends. Only one instance of the object exists. All objects declared at namespace scope (including global namespace) have this storage duration, plus those declared with \cpp{static} or \cpp{extern}.

  \item \textbf{thread} storage duration. The storage for the object is allocated when the thread begins and deallocated when the thread ends. Each thread has its own instance of the object. Only objects declared \cpp{thread_local} have this storage duration.
\end{itemize}

The memory for \emph{dynamic arrays} is allocated ``on the heap'', meaning dynamic arrays have \emph{dynamic storage duration}. Dynamic arrays are not destroyed at
the end of a function or block scope. They are destroyed only by a call to \cpp{delete} or \cpp{delete[]}.


% -------------------------------------------------------------------------------------------------
\subsection{Smart pointers}
We have (static) arrays and dynamic arrays. The difference is that static arrays have a constant size, but automatic storage duration, whereas dynamic arrays
have a dynamic size but also dynamic storage duration. Both can be represented by pointers. What if we want to have a dynamic size, but an automatic
storage duration?

This is possible using the special pointer type \cpp{std::unique_ptr} or \cpp{std::shared_ptr}, sometimes referred to as ``smart pointers''. Both are library
types defined in the header \cpp{<memory>}.

Example:
\begin{minted}{c++}
  #include <memory>
  //...
  std::shared_ptr<int> ip{ new int };
  std::unique_ptr<int[]> vec{ new int[10] };
  // or
  std::shared_ptr<int> ip2 = std::make_shared<int>();        // calls `new` internally
  std::unique_ptr<int[]> vec2 = std::make_unique<int[]>(10);

  std::cout << *ip << vec[5];
\end{minted}
%
So, the definition and usage is very close to classical pointers. But you don't need to call \cpp{delete} to free the memory.

The difference between both types is the criterion when to deallocate the memory:
\begin{itemize}
  \item \cpp{std::shared_ptr} is a smart pointer that retains shared ownership of an object through a pointer. Several \cpp{shared_ptr} objects may own the
  same object. The object is destroyed and its memory deallocated when either of the following happens:
  \begin{itemize}
    \item the last remaining \cpp{shared_ptr} owning the object is destroyed;
    \item the last remaining \cpp{shared_ptr} owning the object is assigned another pointer via \cpp{operator=} or \cpp{reset()}.
  \end{itemize}

  \item \cpp{std::unique_ptr} is a smart pointer that owns and manages another object through a pointer and disposes of that object when the
  \cpp{unique_ptr} goes out of scope. The object is disposed of using the associated deleter when either of the following happens:
  \begin{itemize}
    \item the managing \cpp{unique_ptr} object is destroyed
    \item the managing \cpp{unique_ptr} object is assigned another pointer via \cpp{operator=} or \cpp{reset()}.
  \end{itemize}
\end{itemize}

You can have multiple pointers pointing to the same object, \ie to the same memory location. Since a pointer is a simple type, it can directly be copied
to create a new pointer with the same address:
%
\begin{minted}{c++}
int data = 7;

int* p = &data;
int* q = &data; // second pointer pointing to the same object

int* p2 = p;    // copy of the first pointer creates a new pointer
                // pointing to the same object
\end{minted}
%
The same can be done with \cpp{shared_ptr}. They share the same resource, \ie can point to the same object. Whenever you copy a \cpp{shared_ptr}, an internal
counter is increased that counts how many pointers are pointing to the same object. Only if this counter is zero, the \cpp{shared_ptr} is destroyed at the
end of a scope:
 %
\begin{minted}{c++}
{
  std::shared_ptr<int> p{ new int(7) };
  std::shared_ptr<int> q{ new int(7) }; // independent new pointer to another memory location

  {
    std::shared_ptr<int> p2 = p; // copy of the first pointer creates a new pointer
                                 // pointing to the same object, the internal counter
                                 // is increased

  } // at the end of scope, the second pointer is released -> decrease of the internal
    // counter

  std::shared_ptr<int> p3 = p; // another copy of the first pointer, increases the counter

} // at the end of scope, p and p3 are released, decreasing the counter by 2 -> 0. The
  // memory is finally deallocated.
\end{minted}

\begin{guideline}{Guideline}
  Don’t use explicit \cpp{new}, \cpp{delete}, and owning * pointers, except in rare cases encapsulated inside the implementation of a
  low-level data structure. (Herb Sutter)
\end{guideline}

\section{Namespaces\label{sec:namespace}}
Namespaces allow the programmer to resolve name conflicts. Defining the same name in the same scope results in an error. Namespaces define
named (or unnamed) global scopes.
%
\cppline{namespace NAME { ... }}
%
Symbols declared inside the \cpp{namespace} block are visible only in this block. The \texttt{NAME} of the namespace can be used to explicitly
qualify a function-, class, or variable name inside this namespace. A classical example, that we have seen before, is the standard namespace \cpp{std::}.
The symbol \texttt{::} thereby is the \emph{name-resolution operator} and may be chained to address multiple nested namespaces:
%
\begin{minted}{c++}
  namespace Q {
    namespace V { // original-namespace-definition for V
      void f(); // declaration of Q::V::f
    }
    void V::f() {} // OK
    void V::g() {} // Error: g() is not yet a member of V
    namespace V { // extension-namespace-definition for V
      void g(); // declaration of Q::V::g
    }
  }
  namespace R { // not a enclosing namespace for Q
     void Q::V::g() {} // Error: cannot define Q::V::g inside R
  }
  void Q::V::g() {} // OK: global namespace encloses Q
\end{minted}
%
One can separate declaration and definition of symbols, but only symbols from sub-namespaces can be defined within a namespace. Not just the definition,
but also the access is by the name-resolution operator \texttt{::}.

Namespaces can contain functions, classes, global variables. The outer most surrounding scope is called the \emph{global namespace} and is addressed with
an empty name before the name-resolution operator.

\subsection{\cpp{using} Directive and Declaration}
The statement \cpp{using} allows to import names from another namespace. So, those names can then be used without any namespace qualification, without the
name-resolution operator.

Either, you can import all names from a namespace, by the using-directive \cpp{using namespace NAMESPACE}:
%
\begin{minted}{c++}
namespace scprog {
  void foo() { /*...*/ }
}
int main() {
  using namespace scprog;

  foo(); // Calls foo() from the namespace scprog without scprog::
}
\end{minted}
This is allowed in namespace scope and block scope only. Every name from the import namespace is visible as if it were declared in the nearest enclosing namespace.

This does not import the functions or classes into the current namespace, but just makes the names visible. This is important for name-resolution, e.g.
in function calls.

Note the following:
\begin{itemize}
\item If there is already a local name in the scope declared, the name from the using namespace is hidden by the local name
\item A name from the imported namespace hides a same name in an enclosing namespace of the scope
\item Importing the same name from multiple namespaces into a scope results in a compiler error
\item Importing a name that is the same in the global namespace, results in a compiler error
\end{itemize}

\begin{minted}{c++}
#include <iostream>
int foo() { return 1; }      // (1)

namespace scprog1 {
  int foo() { return 2; }    // (2)
}
namespace scprog2 {
  int foo() { return 3; }    // (3)
  int bar() {
    using namespace scprog1;
    return foo();            // OK: calls (3)
  }
}
int main() {
  using namespace scprog2;
  using namespace scprog1;

  std::cout << bar();
  std::cout << foo(); // error: call of overloaded 'foo()' is ambiguous,
                      // candidates: (1), (2) or (3)
}
\end{minted}

\subsubsection{Namenspace alias}
If you have a complicated or multiple enclosed namespaces, you can introduce a new name (an alias) for this namespace in the local scope.
%
\begin{minted}{c++}
namespace Q {
  namespace V {  // sub-namensraum of Q
    void f() {}; // declaration of Q::V::f
  }
}
void g() {
  using R = Q::V; // alias for the namespace Q::V
  R::f();
}
\end{minted}

\subsubsection{Import of some names from a namespace}
Instead of importing all namespace from a namespace (with \cpp{using namespace N}) one could just declare a local name for a specific name in the namespace:
This allows to not only make a name visible in the current scope, but to actually import that name as if declared in the scope.
%
\begin{minted}{c++}
#include <cmath> // std::sqrt
int main() {
  using std::sqrt;

  sqrt(3.0); // call of std::sqrt without namespace qualification
}
\end{minted}

It is important to understand the difference to the \cpp{using namespace} directive. The \cpp{using} declaration is an actual declaration. So, you can
declare the exact same name twice, but if a name was already declared in the scope and you introduce that same name again (with a different meaning),
the compiler throws an error.

Importing the same name of a function with the same signature from multiple namespace may result in an ambiguous function call.
\begin{minted}{c++}
  namespace B {
    void f(int);
    void f(double);
  }
  namespace C {
    void f(int);
    void f(double);
    void f(char);
  }
  void h() {
    using B::f; // introduces B::f(int), B::f(double)
    using C::f; // introduces C::f(int), C::f(double), and C::f(char)
    f('h');      // calls C::f(char)
    f(1);        // error: B::f(int) or C::f(int)?
    void f(int); // error: f(int) conflicts with C::f(int) and B::f(int)
  }
\end{minted}

\subsection{Argument dependent lookup\label{sec:adl}}
In function call name-lookup, all visible names are inspected. With a using declaration and using directive this list of visible name is extended. But
there is a second way of increasing the list of possible candidates for the overload resolution, by \emph{argument dependent lookup}, or \emph{Koenigs lookup}.

There, the names from another namespace are included in the list of visible functions, of a function parameter is declared in that namespace. So, the
compiler looks also in the namespace of the function arguments.
%
\begin{minted}{c++}
namespace scprog {
  struct A { double x = 1.0; };

  void set(A& a) { a.x = 2.0; }

  A operator+(A, A const&);
}

struct B { double x = 1.0; };

int main()
{
  scprog::A a, b;
  scprog::set(a); // OK: explicit namespace qulification
  set(b);         // OK: ADL

  scprog::operator+(a,b); // OK: explicit namespace qulification
  a + b;                  // OK: ADL

  B x;
  set(x); // ERROR
}
\end{minted}

\begin{example}
For example, to compile \cpp{std::cout << std::endl;}, the compiler performs:
\begin{enumerate}
\item unqualified name lookup for the name \cpp{std}, which finds the declaration of namespace \cpp{std} in the header \cpp{<iostream>}
\item qualified name lookup for the name \cpp{cout}, which finds a variable declaration in the namespace \cpp{std}
\item qualified name lookup for the name \cpp{endl}, which finds a function template declaration in the namespace \cpp{std}
\item argument-dependent lookup for the name \cpp{operator<<}, which finds multiple function template declarations in the namespace \cpp{std}
\end{enumerate}
\end{example}

\todos

\printindex

\bibliographystyle{abbrv}
\bibliography{references}


\end{document}
