\chapter{Classes\label{sec:class}}
Object types or user-defined types are called \emph{classes} in C++ and are introduced with the keywords \cpp{class} or \cpp{struct} (or \cpp{union}).
A class defines a new data type and is a collection of data (attributes, member variables), functions (member functions and class functions),
and \textit{associated} types. The individual parts of a class are called \emph{members}. All members can be qualified with an access restriction
(\cpp{public}, \cpp{protected}, or \cpp{private}).

\begin{example}
The following class generalizes the point/vector struct we have seen in the first lecture. It provides functions for the initialization and for
the access and some data:
\end{example}
\begin{minted}{c++}
  class MyPoint
  {
    using value_type = double;      // Associated type

  public:
    MyPoint() {};                   // Default-Constructor

    MyPoint(double x, double y)     // Constructor
      : x_(x), y_(y)                // Initializer-List
    { }

    void set(double x, double y);   // Member function
    double norm() const;            // (constant) member function

    static int dim() { return dim_; }  // Class-function, or static member function

  private:
    double x_ = 0.0, y_ = 0.0;      // Member variables (with default initialization)
    static const int dim_ = 2;      // Static member variable
  };
\end{minted}


% =================================================================================================
\section{Objects\label{sec:class-object}}
Objects are instances of classes, \ie concrete realizations of a class at runtime. Via an object one gets access to the individual (public) members
(variables, functions) of a class. Each new object allocates new memory for all the class members and fill the initial values by using a constructor.
So, after construction of a new instance of a class, the object is in a valid state.

The following statements declare a new object \texttt{p} or \texttt{q} as an instance of the class \texttt{MyPoint}:
%
\begin{minted}{c++}
  MyPoint p{1.0, 2.0}; // or
  MyPoint q(1.0, 2.0);
\end{minted}
%
This declaration is also a definition, since memory is allocated. The initialization (construction) of that object is done by the constructor of the class.
In this example, the constructor gets two doubles, and sets the values of the member variables \texttt{\_x} and \texttt{\_y}.

The definition as above creates an object with automatic storage duration (\ie the object is created on the ``stack''). An analogy to fundamental types,
the object can also be allocated on the ``heap'':
%
\begin{minted}{c++}
  MyPoint* p = new MyPoint{1.0, 2.0};
  // ...
  delete p;
\end{minted}
The operator \cpp{new} returns a pointer to the allocated memory. At first the memory is allocated for all member variables, then the constructor initializes
the variables. Similar to all other dynamic memory allocation, the object must be deleted manually.

Objects created with the \cpp{new}-operator have to be destroyed with the \cpp{delete}-operator to release the memory. At destruction, whether at the end of
a scope for objects with automatic storage duration, or when calling \cpp{delete}, the \emph{destructor} of the class is called (see below). It is responsible
for any cleanup to do.

\begin{rem}
  Similar to fundamental types and arrays you should not work with raw pointers. Instead, use \emph{smart-pointers} like \cpp{std::unique_ptr} or \cpp{std::shared_ptr}!
\end{rem}

\begin{rem}
  When constructing an object using a constructor without arguments, one can either use empty curly braces, \ie
  %
  \cppline{MyPoint p{};}
  %
  or omit any braces, \ie
  %
  \cppline{MyPoint p;}
  %

  It is, contrariwise, not allowed to use round empty braces \cpp{MyPoint p()}. This would declare a function \texttt{p} returning a \texttt{MyPoint}, instead.
\end{rem}


% =================================================================================================
\section{Access to class members}
To access the data of a constructed class object, one can either use the \textit{dot}-access operator \texttt{a.b},
or the \textit{pointer}-access operator \texttt{a->b}. The former can be used on expressions of \emph{class type} and the latter
or \emph{pointer to class type}. Static members or properties of the class can be accessed using the name-resolution operator \texttt{A::B}.

Examples:
\begin{minted}{c++}
  MyPoint  p1{1.0, 2.0};
  MyPoint* p2 = new MyPoint(1.0, 2.0);
  std::unique_ptr<MyPoint> p3 { new MyPoint(1.0, 2.0) };

  double n = p1.norm();   // access on class type
  p2->set(2.0, 1.0);      // access on pointer to class type
  double n2 = p3->norm(); // access on pointer to class type

  int dimension = MyPoint::dim(); // access of static members

  delete p2;              // raw pointers must be deleted
\end{minted}

\begin{rem}
  The access functions \cpp{.}, \cpp{->}, and \cpp{::} are operators in C++ and thus have a precedence and an associativity.
\end{rem}

% =================================================================================================
\section{Class Data\label{sec:class-attributes}}
Attributes of a class are declared in a class definition and include \emph{methods}, \emph{variables}, \emph{types}, and even sub-classes or \emph{types}.

Each instance of a class gets its own copy of all attributes, except for static attributes. Those exist exactly once for each class type and are bound
to the type and not to the instance.

% -------------------------------------------------------------------------------------------------
\subsection{Member variables}
Variables, declared in a class, can be used to provide data that is distinct in each instance of a class. If a variable is qualified with \cpp{const} or \cpp{&},
it must be initialized directly in each constructor, since const and reference require direct initialization.
%
\begin{minted}{c++}
  class MyPoint
  {
  public:
    MyPoint(...) : ..., dim(2) {}
  //...
    double x;
    double y;
    bool const valid = true;
    int const dim;
  };
\end{minted}

If member variables are supposed to be initialized with a default value in all constructors if not specified otherwise, the initial value can be
prescribed directly on declaration, like \cpp{bool const value = true;}

% -------------------------------------------------------------------------------------------------
\subsection{Methods}
Methods or member functions of a class are functions defined in the scope of the class that have access to all member variables of the instance of that class.
Member functions can be defined inline or outside of the class:
%
\begin{minted}{c++}
  // in header file:
  class MyPoint
  {
  //...
    void set(double x, double y) // inline definition
    {
      x_ = x;
      y_ = y;
    }

    double norm() const;  // just declaration
  };

  // in .cpp-file
  double MyPoint::norm() const
  {
    return std::sqrt(x_*x_ + y_*y_);
  }
\end{minted}

If you intend to include the header file with the class definition multiple times, you need to define all functions in a source file, \ie a separate
translation unit. This prevents from multiple definitions of the same function. Functions defined directly in the class automatically are classified as
\cpp{inline} functions and thus can exist multiple times.

Within a class method, one gets direct access to all class members (functions, variables, types...). Additionally, there is a special pointer, \cpp{this},
pointing to the current instance of that class in memory. This can be used to access the members as well (see pointer-access operator):
%
\begin{minted}{c++}
  double MyPoint::norm() const
  {
    return std::sqrt(x_*x_ + y_*y_);
  }
  // or
  double MyPoint::norm() const
  {
    return std::sqrt(this->x_*this->x_ + this->y_*this->y_);
  }
\end{minted}

The keyword \cpp{const} in the member function declaration means, that 1. no member of the class is changed and 2. only other \cpp{const} functions are called
in that function. If you have a \cpp{const} class object or a \cpp{const&} to a class object, only \cpp{const} member functions can be called:
%
\begin{minted}{c++}
  MyPoint const& q = p;
  q.set(3.0, 4.0);               // ERROR
  std::cout << q.norm() << "\n"; // OK, norm is const-function
\end{minted}

% -------------------------------------------------------------------------------------------------
\subsection{Static Members}
Static variables (class variables) or static function (class functions) are shared by all instances of that class and are introduced with the keyword \cpp{static}.
Static variables must be defined outside of the class. Since they are independent of any instance. Static members can not be initialized in a constructor.
%
\begin{minted}{c++}
  class MyPoint
  {
  public:
    static int nPoints;
  };
  ...
  int MyPoint::nPoints = 0;
\end{minted}

Since \cxx{17}\marginpar{[\cxx{17}]}, static variables can be marked \cpp{inline} that allows to initialize them inside the class, like regular member
variables, \eg \cpp{inline static int nPoints = 42;}

Static variables can have the qualifier \cpp{const} or \cpp{constexpr}. In this way one can define constant properties of a class. If the static
constant is additionally an integral type it can always be defined in the class directly:
%
\begin{minted}{c++}
  class MyPoint
  {
  public:
    static const int dim_ = 2;
  };
\end{minted}

Static members can be accessed by the class name and the name resolution operator \texttt{::}. Within any other class member function or static function in the
class this can be omitted.

\begin{rem}
  The definition of a static class member variable must be put into a source file (not in a header file). Otherwise it would be defined multiple times.
\end{rem}

Also functions can have the \cpp{static} keyword, indicating that they are bound to the class type and not to any instance of the class. Within static
function, only static member variables of the class can be accessed:
%
\begin{minted}{c++}
  // in header file
  class MyPoint
  {
  public:
    static int dim() { return dim_; }
    static int dim_;
  };
  ...
  // in source file
  int MyPoint::dim_ = 2;
  ...
  int main() {
    std::cout << MyPoint::dim();
  }
\end{minted}

Static member function must be defined inline in the class and are automatically \cpp{inline}.

% -------------------------------------------------------------------------------------------------
\subsection{Class member types / associated types}
Within a class, local typedefs or type aliases can be introduced. This allows to have a central place in the class definition where to fix types used
somewhere in the class. And it allows to give that information to the user of the class.

We call those types \emph{associated types} since they are associated with the class type (and thus shared by all instances of that class). Types
are introduced using the keyword \cpp{typedef} or \cpp{using}:
%
\begin{minted}{c++}
  class MyPoint
  {
  public:
    typedef int     size_type; // old syntax
    using value_type = double; // C++11 syntax
  };
\end{minted}
(where I would prefer the new syntax)

In containers of the standard library (and of many other libraries) some generic type names are used:
\begin{itemize}
  \item \cpp{size_type} An integer type that can be used for the size of a container (and for indices)
  \item \cpp{value_type} The type of the elements stored in a container
  \item \cpp{reference}, \cpp{pointer}, \cpp{iterator}, \cpp{const_iterator} Types used to defined the return type of access methods and iteration methods.
\end{itemize}
%
\begin{minted}{c++}
  class MyPoint
  {
  public:
    using size_type  = int;
    using value_type = double;

    value_type& operator[](size_type i) { return i == 0 ? x : y; }

  private:
    value_type x, y;
  };
\end{minted}

Types can be accessed like other static properties, using the name resolution operator: \cpp{MyPoint::value_type}.


% =================================================================================================
\section{Constructors\label{sec:constructor}}
Constructors are special member functions with its name equal to the class name and no return type. Constructors initialize an object of a class type
and create the workspace for all other members. Each instantiation of a class (directly, or with \cpp{new}), with or without arguments, calls a constructor.
If no constructor is defined at all, the compiler tries to generate one automatically with default behavior.

\begin{example}
  A constructor could open a file, where other member function tend to write into. At the end of the lifetime of the class this file must be closed.
\end{example}

It is allowed to declare multiple constructors, while following the rules of regular function overloading, \ie the constructors must have different signatures.

The construction of a class happens in three stages: 1. memory for all member variables (and all base-classes) is allocated, 2. all members (and base-classes) are
initialized by the constructor initializer-lists, 3. the constructor body is executed to finalize any initialization.

% -------------------------------------------------------------------------------------------------
\subsection{Initializer Lists}
Constructors are responsible for initializing all member variables. The initialization may invoke the corresponding constructors of those variables. Since all
members should be initialized with their corresponding constructors, a special syntax is reserved for this initialization: constructor \emph{initializer lists}.

Directly after the constructor argument list (after the closing round bracket), a colon \texttt{:} can introduce the initializer-list. It is a comma-separated
list of initializer expressions (constructor calls) for each member variable, base-class, or a call to another constructor (\emph{constructor chaining}, or
\emph{delegation}).

After the initializer list, the regular constructor function body must be follow.

Example:
\begin{minted}{c++}
class MyPoint
{
public:
  MyPoint(double x, double y) : x_{x}, y_(y) { }  // variable initializers
  MyPoint() : MyPoint(0.0, 0.0) { }               // delegating constructor
  double x_,y_;
};

class MyPoint2
{
public:
  MyPoint2(double x, double y) : point_{x,y} { }

  MyPoint point_;
}
\end{minted}
%
In the first class \texttt{MyPoint}, two member variables \texttt{x\_} and \texttt{y\_} are initializes with the values of the constructor arguments. Thereby,
one can use the initialization brackets similar to regular variables, i.\,e.\ curly or round brackets. The second constructor delegates the initialization to
the first constructor, by calling the class-constructor with some fixed arguments. The second class \texttt{MyPoint2} has a variable of user-defined type
that is also initialized by a constructor call passing two arguments.

\textbf{Note}: The order of the initialization should follow the order of the declaration of the member variables in the class (and is independent of the order
of the constructor arguments).

\textbf{Note}: If some member variables have default initializers, like \cpp{int dim = 2;} those initializations are appended to \underline{all} constructor
initializer lists, if no other value for that variable is prescribed.


% -------------------------------------------------------------------------------------------------
\subsection{Some special constructors}
Some constructors are special, either because they could be generated automatically by the compiler, or because of the way those constructors are invoked. We
will discuss here the default constructor, copy constructor, move constructor, and the destructor (that is not directly a constructor, but kind of the opposite).


\subsubsection{Default constructor}
A constructor that is defined with an empty parameter list is called the \emph{default constructor}. It is used to bring the class into a default valid state
by (typically) initializing all member variables with its default values.

To use the default constructor when instantiating a class there are two variants:
%
\begin{minted}{c++}
// without any argumentlist
MyPoint p1;
MyPoint* q1 = new MyPoint;
// with "universal" initialization
MyPoint p2{};
MyPoint* q2 = new MyPoint{};
\end{minted}

If there is no other constructor defined, the compiler automatically generates a default constructor with default behavior, \ie that calls the default constructor
on each member variable (and base-class). For integral can floating-point types the default initializer value is \texttt{0}, if not specified otherwise by
putting the value in the variable definition, \ie \cpp{int dim = 2;}.

If there is any other constructor defined, one can force the compiler to generate the default constructor automatically, by adding the keyword \cpp{= default} to
the declaration of the constructor:
%
\begin{minted}{c++}
class MyPoint
{
public:
  MyPoint() = default; // initializes x,y with default values
  MyPoint(double x_, double y_) { ... } // another user-defined constructor
  double x,y;
};
\end{minted}


\subsubsection{Copy constructor}
A \emph{copy constructr} is used to create a new instance as a copy of another object. This constructors expects exactly one argument, an const lvalue-reference
to the class-type itself:
%
\begin{minted}{c++}
class MyPoint
{
public:
  MyPoint(MyPoint const& that) // copy constructor
    : x(that.x)
    , y(that.y)
  { }
  double x,y;
};
\end{minted}
%
In this constructor, it is recommended to also use the constructor initializer lists.

Copy constructors are used/invoked whenever:
\begin{itemize}
  \item An object is passed by value to a function:
    \cppline{double norm(MyPoint p);}
  \item An object is returned by value from a function:
    \cppline{MyPoint generate_point();}
    (Maybe return value optimization or copy-elision kicks in)
  \item An object of class-type is passed as single argument to another constructor:
    \begin{minted}{c++}
  MyPoint p(1.0, 2.0);
  MyPoint p2(p);
  MyPoint p3 = p;
    \end{minted}
  \item In initializers lists:
    \begin{minted}{c++}
  class A {
    A(MyPoint const& p_) : p(p_) {}
    MyPoint p;
  };
    \end{minted}
\end{itemize}

The default behavior of a copy constructor is, that each member is copied (using its copy constructor). This default behavior is automatically generated
by the compiler if the compiler can be sure that this is the expected behavior. This is not the case if
\begin{itemize}
  \item any member (or base-class) is non-copyable
  \item a user-defined move-constructor is provided
  \item a user-defined move assignment operator is provided
\end{itemize}

If the default behavior at least would be will defined (\ie all member can be copied), one can force the compiler to generate the default copy constructor
by using the keyword \cpp{= default}, as above:
%
\begin{minted}{c++}
  class MyPoint
  {
  public:
    MyPoint(MyPoint const& that) = default;
    double x,y;
  };
\end{minted}

Also, one could explicitly mark the copy constructor as \emph{deleted} to enforce that a class can not be copied. This can be accomplished by using the
keyword \cpp{= delete}, similar to the default.

\begin{guideline}{Attention}
  While the arguments to the copy constructor might be given as const or non-const lvalue reference, it is not allowed to pass the argument by value, \eg
  \cpp{MyPoint(MyPoint p)}. Why not?
\end{guideline}

The cousin of the copy-constructor is the \emph{copy assignment operator}. It is used to copy the content of an object to another already existing object.
The argument to the assignment operator thereby is the same as for the copy constructor, but it has to return a reference to the class type:
%
\begin{minted}{c++}
  class MyPoint
  {
  public:
    MyPoint(MyPoint const& that) = default; // copy constructor

    MyPoint& operator=(MyPoint const& that)
    {
      x = that.x;
      y = that.y;
      return *this;
    }
    double x,y;
  };
\end{minted}

\textbf{Note}: You can not use the initializer-lists anywhere outside of constructors, thus especially not in assignment operators. Therefore, the copy has to
move to the function body.

\begin{guideline}{Principle}
  If you define a copy-constructor you should also define a copy assignment operator.
\end{guideline}


\subsubsection{Move constructor}
A \emph{move constructor}\marginpar{[\cxx{11}]} is used to create a new instance of an object, by ``steeling'' the data from another object,
leaving it in an undetermined state. This requires that the moved-from object is not used anymore. The concept of such a state is a temporary, or
an rvalue. So, move constructors expect as single argument an \underline{rvalue-reference} to the class-type: \cpp{MyPoint(MyPoint&& that)}.

Move constructors are introduced in \cxx{11} (together with rvalue reference) to allow the fast construction of copies of objects, that are destroyed anyway.
Thus, one just use the data ob the temporary object, without copying it. This can be motivated by a class storing a pointer to a large dynamic array
that was allocated in some constructor and would be destroyed in a destructor. Instead of copying all the elements of that array, a move-constructor
could just copy the pointer to the array and set it to \cpp{nullptr} inside the moved-from object to forbid the deallocation:
%
\begin{minted}{c++}
class MyPoint
{
public:
  MyPoint()                   // default constructor
    : big_data(new double[1000000])
  {}

  ~MyPoint()                  // destructor (see below)
  {
    delete [] big_data;       // deallocate the data (if not nullptr)
  }

  MyPoint(MyPoint&& that)     // move constructor
    : big_data(that.big_data) // copy just the pointer
  {
    that.big_data = nullptr;  // unset the point in `that`
  }

  double* big_data;
};
\end{minted}
%
In contrast to the copy constructor, the move constructor expects a mutable rvalue-reference, so that fields can be made invalid.

Move constructors are used/invoked when:
\begin{itemize}
  \item initialization: \cpp{T a = std::move(b);} or \cpp{T a{std::move(b)};}, where \texttt{b} is of type \texttt{T};
  \item function argument passing: \cpp{f(std::move(a));}, where \texttt{a} is of type \texttt{T} and \texttt{f} is \cpp{void f(T t)};
  \item function return: return \texttt{a}; inside a function such as \cpp{T f()}, where \texttt{a} is of type \texttt{T} which has a move constructor.
\end{itemize}

The default behavior of a move-constructor is to move construct all its members from the members of the passed argument. This default move constructor
can be generated by the compiler automatically. This does not always work, especially if you have data managed dynamically. Thus, a move constructor is
implicitly declared only of
\begin{itemize}
  \item there are no user-declared copy constructors;
  \item there are no user-declared copy assignment operators;
  \item there are no user-declared move assignment operators;
  \item there are no user-declared destructors;
\end{itemize}

and is implicitly deleted if some dana members (or base-classes) can not be moved.

In case the default behavior would make sense but another constructor or destructor of the list above is defined, one can force the compiler to generate
the move-constructor automatically, using the keyword \cpp{= default}, as above.

Also, one could explicitly mark the move constructor as \emph{deleted} to enforce that a class can not be moved. This can be accomplished by using the
keyword \cpp{= delete}, similar to the default.

Similar to the copy constructor, there is a cousin of the move constructor, the move assignment operator: This should be defined whenever a move constructor
is provided:
\cppline{MyPoint& operator=(MyPoint&& that)}


% -------------------------------------------------------------------------------------------------
\subsection{Destructor\label{sec:destructor}}
A destructor is kind of the opposite to a construct. It is responsible for releasing all resources allocated in any other constructor of that class, \eg
to free all memory or to close opened files. A destructor is called whenever the life-time of a class end, either at the end of a scope or explicitly via
the \cpp{delete} operator.

Destructors are indicated using a \textasciitilde in front of the class name. They have an empty parameter list and no return type:
%
\begin{minted}{c++}
class MyPoint
{
public:
  MyPoint(int num = 99999) : big_data(new double[num]) { }
  ~MyPoint() { delete[] big_data; }   // destructor, frees the memory
private:
  double* big_data = nullptr;
};
\end{minted}
At the end of a user-defined destructor, the destructors of all member variables are called automatically.

The default behavior of a destructor is, that simply all members are destroyed using their destructors. This default behavior is automatically generated if
no user-defined destructor is provided. If any member cannot be destructed, the implicitly generate destructor is deleted.

% =================================================================================================
\section{Rule of five / Rule of zero}

Since the compiler sometimes generates constructors (and destructor) automatically and in some combinations not, there is a rule that is known as
\emph{rule of five} stating that if you write any of copy constructor, copy assignment operator, move constructor, move assignment operator, or destructor,
you have to implement them all.

Especially for copy or move assignment operators, you have to be careful not to assign an object to yourself. Or you have to handle this case manually. The
destructor needs to be aware of the move operations and you need to be careful when an error happens somewhere in the process of construction or destruction.

This is often very complicated and thus a common guideline is, to write you class in such a ways the the compiler always generates everything for you and you
don't need to write any of the special constructor (including the default constructor). This pattern is called \emph{rule of zero} and means, especially,
that you should not do dynamic memory management inside of your classes (or just in one class that is responsible just for this), but use library types that
are good replacements for your manual management, like \cpp{std::shared_ptr}, \cpp{std::unique_ptr}, \cpp{std::vector}, \cpp{std::array}, or \cpp{std::string}.

\begin{guideline}{Rule of Zero}
  Use Data-members that already have an own copy/move constructors and assignment operators and destructor for their data-management. So, you are not
  responsible for the release of allocated memory and don't need to implement an own constructor and destructor yourself.
\end{guideline}

\begin{figure}[ht]
\begin{center}
\includegraphics[width=.8\textwidth]{images/auto_constructors.png}
\caption{Howard Hinnant: \emph{Everything You Ever Wanted To Know About Move Semantics (and then some)} (ACCU 2014)}
\end{center}
\end{figure}

% =================================================================================================
\section{Access restriction\label{sec:access-restriction}}
In C++ it is possible to restrict in classes who can access the members of that class. There are two access groups:
\begin{itemize}
	\item \cpp{public}: Everyone can access the members.
	\item \cpp{private}: Only other class members are allowed to access this attribute.
\end{itemize}
Additionally, there is the access restriction \cpp{protected} that we will see later.

\begin{example}
  For the illustration of access restrictions, consider the following example:
  \begin{minted}{c++}
  class Point {
  public:
    double x,y;
    void foo() {
      a = 2.0;  // OK: in a member function of the class we can access private data
      bar();    // OK: in a member function of the class we can call private methods
    }
  private:
    double a;
    void bar();
  };
  // ...
  Point p{1.0, 2.0}; // Initialization

  p.x;   // OK: x is a public member variable
  p.a:   // ERROR: a ist a private member and can not be accessed from outside the class

  p.foo(); // OK: foo is a public member function
  p.bar(); // ERROR: bar is a private member function
  \end{minted}
\end{example}

The keywords \cpp{public}, \cpp{private} (and \cpp{protected}) followed by a colon \texttt{:} introduces a block in which all following members
have this access restriction.

\begin{rem}
  In classes introduced with \cpp{class}, the default access restriction is \cpp{private}, whereas in \cpp{struct}, the default access restriction is
  \cpp{public}.
\end{rem}

\begin{rem}
  Also constructors can be restricted. This allows to instantiate a class (or copy/move a class) in a specific way only. If there are no public
  constructors available, the class can not be instantiated directly and one has to provide a ``factory''-function instead, \ie a static public
  member function that constructs and returns the instantiated class.
\end{rem}


% =================================================================================================
\section{Inheritance\label{sec:inheritance}}
(The chapter inheritance is touched here only slightly, see other lectures about object oriented design for more details)

Classes characterize properties and methods of a specific space of objects. An example is a class representing a vector space. Instances of that
class are vectors. But, there are other spaces, with more or less properties, like a normed vector space that additionally has a norm function, or
a scalar-product space that additionally has a scalar-product. So, we can find a hierarchy of classes with a general class and more and more specialized
classes. The more specialized ones can inherit properties / members of the more general class, like a normed vector space inherits all methods from a
vector space. This principle is call inheritance in C++.

In the head of a class, one can list other classes we want to inherit properties and members from. Thereby, we can specify how those inherited members
are visible (can be accessed) (Either they are just properties we want to use internally: \cpp{private}, or all properties are made visible to the outside:
\cpp{public}).

\begin{minted}{c++}
class VectorSpace {                     // Base class
public:
  VectorSpace();
  VectorSpace operator+(VectorSpace const& rhs) const;
  VectorSpace operator-(VectorSpace const& rhs) const;
  // ...

  int dimension_;                       // is public to the outside and derived classes

protected:
  double component(int i);              // is accessible from derived classes, but not public

private:
  std::vector<double> components_;      // is private to this class
};

class NormedVectorSpace : public VectorSpace
{
public:
  double norm() const
  {
    double n = 0.0;
    for (int i = 0; i < dimension_; ++i)  // we can access public members of base class
      n += component(i)*component(i);     // we can access protected functions of base class
    return std::sqrt(n);
  }
};
// ...
VectorSpace vec1;
vec1.dimension_     // OK: public member
vec1.component()    // ERROR: protected members not accessible
vec1.components_    // ERROR: private members not accessible
vec1.norm()         // ERROR: VectorSpace has no method norm()

NormedVectorSpace vec2;
vec2.dimension_     // OK: public derived member
vec2.component()    // ERROR: protected member of VectorSpace
vec2.norm()         // OK: public function
\end{minted}

The derived class can call protected and public members of the base class, and can even call its constructors.

\begin{rem}
  Again, the difference between \cpp{class} and \cpp{struct} is that for class the inheritance is \cpp{private} by default and for struct
  it is \cpp{public} by default.
\end{rem}
