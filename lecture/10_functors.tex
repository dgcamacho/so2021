\chapter{Functors}
A Functor is an abstraction of a classical function, but implemented as class type. It can be used (called) like a regular function, but has additional
advantage to use all the flexibility of a class, like local member variables, or other member functions that can be called.

The basic idea is, that a class that implements the \cpp{operator()} can be evaluated like a function. The call to that special member function has the same
syntax as a regular function invocation.

Example:
\begin{minted}{c++}
  struct Square {
    // Implementation of the bracket operator
    double operator()(double x) const {
      return x*x;
    }
  };

  int main() {
    // Instantiation of the class
    Square square{};

    // Use the instance of the class, like a regular function
    double y = square(3.0); // = 9
  }
\end{minted}

\begin{guideline}{Guideline}
  Implement functors as classes with an overloaded \cpp{operator()} as \cpp{public} and (often) \cpp{const} member-function.
\end{guideline}

% =================================================================================================
\section{Functors as function parameters}
Since functors are regular user-defined types, they can be passed to any other function as function parameter. This allows to write algorithms
without specifying the concrete operation explicitly in code, but expecting this operation to be passed as argument. Examples for this type
of algorithms are sorting algorithms, where your want to specify the comparison operator, \eg \cpp{std::less}, as argument so that you
can easily switch between increasing and decreasing order of the elements.

Another example is the function \cpp{for_each}, that applies to each element in a container a function (\eg to change the value of the element):
\begin{minted}{c++}
  void for_each(std::vector<double>& vec, Square const& f) {
    for (auto& v_i : vec)
      v_i = f(v_i);
  }

  int main() {
    std::vector<double> v{1.0, 2.0, 3.0, 4.0, 5.0};

    // Instantiation of the class
    Square square{};

    // Pass the functor to the function
    for_each(v, square);

    // -> {1.0, 4.0, 9.0, 16.0, 25.0}
  }
\end{minted}
%
We could even instantiate the functor directly in the call of the \cpp{for_each} function, \ie
\cppline{for_each(v, Square{});}

The problem with this approach is, that we might want to pass different functors to the same function / algorithm. This is a perfect example
of a function template. Instead of explicitly specifying the type of the functor that we can pass, we specify a template parameter for the functor
argument:
\begin{minted}{c++}
  template <class F>
  void for_each(std::vector<double>& vec, F const& f) {
    for (auto& v_i : vec)
      v_i = f(v_i);
  }
\end{minted}

This allows to call the function \cpp{for_each} the same way as before (thanks to ATD), but we can also pass another functor without reimplementing the
algorithm.

\begin{rem}
  A more generic implementation of the \cpp{for_each} function uses a second template parameter for the container type, \ie
  %
  \begin{minted}{c++}
  template <class Range, class F>
  void for_each(Range& range, F const& f) {
    for (auto& element : range)
      element = f(element);
  }
  \end{minted}
  %
  This allows to use that function with \cpp{std::vector<...>} of any type supported by the functor, or \cpp{std::list}, \cpp{std::array}, or
  any other type that can be iterated over using range-based for-loops.
\end{rem}


% ==============================================================================
\section{Parametrized Functors}
We can not only parametrize the functions that use the functor, but also the functor itself. The allows to specify the argument or return types
of the \cpp{operator()} implemented inside the functor, or any other useful type that we need to implement the functor operation.

\begin{example}
  The square functor from above could be applied to any number type that supports multiplication. Thus, we could make the implementation more
  generic by parametrizing the functor:

  \begin{minted}{c++}
  template <typename domain_type, typename range_type = domain_type>
  struct Square
  {
    range_type operator()(domain_type x) const {
      return x*x;
    }
  };
  \end{minted}

  In the instantiation of the class, we now have to instantiate the functor class template as well:
  \cppline{for_each(v, Square<double>{});}
\end{example}

Often, the functor class itself needs no parametrization, but the member function \cpp{operator()} depends on template parameters. So,
one could directly parametrize just the member function, instead of the whole class:

Example:
\begin{minted}{c++}
  struct Square
  {
    template <typename domain_type>
    auto operator()(domain_type x) const {
      return x*x;
    }
  };
\end{minted}
%
For simplicity, I have used the return type \cpp{auto} to let the compiler automatically deduce the return type from the expression in the
function. The \cpp{auto} return type gets very useful when combined with input template parameters.

The call to \cpp{for_each} is now as simple as before:
\cppline{for_each(v, Square{});}
and we do not need to know the type of the argument in advance.


% =================================================================================================
\section{Lambda expressions}
In \marginpar{[\cxx{11}]}\cxx{11} the definition of functors was simplified a lot. Instead of writing a class or class template first, then instantiating that
class and passing it to an algorithm, one can define an ``anonymous'' functor directly in the line of usage. The compiler then generates automatically
a corresponding class with a unique names, instantiates that class and passes it to the function.

Those anonymous functors are called \emph{lambda expressions} in C++ and can be declared as follows:
%
\begin{minted}{c++}
  [captures...] (parameters...) -> ReturnType { function body };
  /* or shorted in case of empty parameter list */
  [captures...] { function body };
\end{minted}
%
Without the \emph{capture} clause, it looks like a regular function definition.
\begin{itemize}
  \item The \emph{captures} is a list of zero or more variables or references that can be used inside the function.
  \item \emph{parameters} is a list of function parameters including its types and qualifiers (like in regular function declarations)
  \item \emph{ReturnType} is the (optional) (trailing) return type of the function, that can depend on the types of the function parameters
        It can be omitted, resulting in the automatic return type deduction, like in functions returning \cpp{auto}.
  \item The function body may contain any sequence of C++ statements and (optionally) a return statement.
\end{itemize}

\begin{example}
  The following example illustrates the usage of lambda expressions, compared to classical functors:
  \begin{minted}{c++}
  // functor with internal member variable to store data.
  struct Scale {
    double factor_;

    Scale(double factor = 1.0)
      : factor_(factor)
    {}

    double operator()(double value) const
    {
      return value * factor_;
    }
  };

  int main() {
    std::vector<double> v = {1.0, 2.0, 3.0, 4.0, 5.0};

    double factor = 3.0;

    // Instantiate a regular functor with constructor argument
    for_each(v, Scale{factor});

    // Use a lambda expression
    for_each(v, [factor](double value) { return factor * value; });
  }
  \end{minted}
  %
  The variable \texttt{factor} is captured, and can thus be used inside the lambda expression.
\end{example}

The capture list is very similar to a list of constructor argument of the functor. All variables in that list introduce a local
member variable, that can be used inside the lambda expressions function body. Any other variable cannot be used there, since it
generates an unnamed functor class with an \cpp{operator()} member function that is called on evaluation.

Using the capture list, a connection to the surrounding scope can be established. The following values are possible in the capture list:
\begin{minted}{c++}
[]    // no capture
[a]   // copies the variable a into a local member variable of the functor
[&a]  // stores a reference to the variable a
[=]   // copies all variables of the surrounding scope that are used inside the function body
[&]   // references all variables of the surrounding scope ...
\end{minted}


\subsection{Generic lambdas}
Like functions and classes, also lambda expressions can be made generic. In the example \texttt{Square} above, we have parametrized the
inner member function \cpp{operator()} with the type of the arguments. A similar templated bracket operator for lambda expressions
can be obtained, by using the placeholder \cpp{auto} inside the function parameter list:
%
\begin{minted}{c++}
  for_each(v, [factor](auto value) { return factor * value; });
\end{minted}
%
This introduces a member-function template with an unnamed template parameter, \ie it is similar to the functor definition
%
\begin{minted}{c++}
  template <class T>
  struct Scale {
    T factor_;

    Scale(T factor = T(1))
      : factor_(factor)
    {}

    template <typename S>
    auto operator()(S value) const
    {
      return value * factor_;
    }
  };
\end{minted}
%
but, we do not have a name for the type of the parameter like \texttt{S} in the explicit functor definition.

\begin{rem}
  The type of the function parameters in generic lambdas can be obtained by using the \cpp{decltype} operator.
\end{rem}

Similar to classical templates and the \cpp{auto} placeholder in variable declarations, the function parameters in generic lambdas
can be qualified with \cpp{const}, reference and pointer. But they cannot be used as placeholders in other templated types, \eg
\cpp{std::vector<auto>} is not allowed as function parameter in generic lambdas.

\begin{rem}
  With \marginpar{[\cxx{20}]}\cxx{20} one can introduce named template parameters in generic lambdas. Therefore, one adds a declaration
  of template parameters after the capture block in the lambda definition:
  %
  \begin{minted}{c++}
  [...] <class P1, class P2, (...)> (P1 param1, P2 param2...) -> ReturnType { function body };
  \end{minted}
\end{rem}


% =================================================================================================
\section{Abbreviated function template}
Since \marginpar{[\cxx{20}]}\cxx{20}, a similar syntax as used for generic lambda can be used for regular function. When placeholder types \cpp{auto} appear in the
parameter list of a function declaration or of a function template declaration, the declaration declares a function template, and one
invented template parameter for each placeholder is appended to the template parameter list.
%
\begin{minted}{c++}
  void f1(auto arg); // same as template<class T> void f1(T arg)
  void f2(auto const& arg); // same as template<class T> void f2(T const& arg)
\end{minted}
%
This is especially useful if the actual type of the argument is irrelevant but the algorithm allows to generically pass arguments of different
type.
