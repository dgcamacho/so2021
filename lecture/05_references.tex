
\section{References and Pointers\label{sec:references}}
Although used already, the references need a revisit. References can be understood as alias to (existing) objects. Compared to classical
pointer, they do not represent the address of the references object, but the data of the objects directly.

A reference can be declared like a regular variable, using the reference qualifier \texttt{\&}:
\begin{minted}{c++}
TYPE & var1  = aliased_objekt; // (1)
TYPE && var2 = aliased_objekt; // (2) ... since C++11
\end{minted}
Where (1) is called a lvalue-reference \index{Reference!lvalue reference} and (2) a rvalue-reference \index{Reference!rvalue reference}. A reference
must be initialized directly and since it is not possible to change where reference refers to.

\begin{minted}{c++}
int i = 0;
int& r = i; // r references i
r = 1;      // changes the value of i => i == 1
i = 2;      // also r == 2
\end{minted}

References itself are no objects and do not need own memory. That is why there are no arrays of references and no references of references!
\begin{minted}{c++}
int& a[3]; // error
int& &r;   // error
\end{minted}


% -------------------------------------------------------------------------------------------------
\subsection{lvalues and rvalues / value categories}
See \url{https://en.cppreference.com/w/cpp/language/value_category}:

Each C++ expression (an operator with its operands, a literal, a variable name, etc.) is characterized by two independent properties:
a \emph{type} and a \emph{value category}. Each expression has some non-reference type, and each expression belongs to exactly one of the three
primary value categories: \emph{prvalue}, \emph{xvalue}, and \emph{lvalue}.
\begin{itemize}
  \item a \emph{glvalue} (``generalized'' lvalue) is an expression whose evaluation determines the identity of an object, bit-field, or function;
  \item a \emph{prvalue} (``pure'' rvalue) is an expression whose evaluation either \begin{itemize}
      \item computes the value of the operand of an operator (such prvalue has no result object), or
      \item initializes an object or a bit-field (such prvalue is said to have a result object). All class and array prvalues have a result object
            even if it is discarded. In certain contexts, temporary materialization occurs to create a temporary as the result object;
  \end{itemize}
  \item an \emph{xvalue} (an ``eXpiring'' value) is a \emph{glvalue} that denotes an object or bit-field whose resources can be reused;
  \item an \emph{lvalue} (so-called, historically, because lvalues could appear on the left-hand side of an assignment expression) is a
        \emph{glvalue} that is not an \emph{xvalue};
  \item an \emph{rvalue} (so-called, historically, because rvalues could appear on the right-hand side of an assignment expression) is a
        \emph{prvalue} or an \emph{xvalue}.
\end{itemize}

\begin{figure}[ht]
\begin{center}
\includegraphics[width=.5\textwidth]{images/value_categories}
\end{center}
\end{figure}

This categorization results from the idea that every expression can be characterized by two orthogonal properties:
\begin{enumerate}
\item[i] \textbf{has identity}: Object that has an address, a pointer, the user can determine whether 2 copies are identical
\item[m] \textbf{can be moved from}: We are allowed to leave the source of a ``copy'' in some indetermined, but valid, state.
\item[$\sim$] has not the property.
\end{enumerate}

The combination \texttt{($\sim$i)($\sim$m)} does not really exists and thus, we have three leaf categories, see Figure.

Examples for lvalues, rvalues, and prvalues:
\begin{minted}{c++}
int i = 3;
int j = 4;

i              // the name of a variable is an lvalue
123, bool, 'c' // literals (except for string literals) are prvalues

// a + b, a % b, a & b, a << b, and all other built-in arithmetic expressions are prvalues

int c = i * j; // OK, (p)rvalue on the right-hand side of an assignment
i * j = 42;    // ERROR, rvalue not allowed on the left-hand side of an assignment

// more on lvalues:
i = 43;           // OK, i is an lvalue
int* p = &i;      // OK, i is an lvalue

int& foo();       // a function call, whose return type is lvalue reference is an lvalue
foo() = 42;       // OK, foo() is an lvalue
int* p1 = &foo(); // OK, foo() is an lvalue

// more on rvalues:
int foobar();        // a function call, whose return type is non-reference is an prvalue
int j = 0;
j = foobar();        // OK, foobar() is an rvalue
int* p2 = &foobar(); // ERROR, cannot take the address of an rvalue
j = 42;              // OK, 42 is an rvalue
\end{minted}


% -------------------------------------------------------------------------------------------------
\subsection{lvalue- and rvalue references}
We distinguish references to lvalues and rvalues as lvalue-references and rvalue-references, denoted by one or two \texttt{\&}, respectively. So,
lvalue-references refer to existing objects with an identity, \ie lvalues, whereas rvalues typically refer to something that goes out of scope, who's
lifetime ends. Rvalue references can be used to extend the lifetimes of temporary objects:

\begin{tabular}{p{0.35\textwidth}|p{0.65\textwidth}}
\begin{minted}{c++}
int f() { return 42; }
int& g(int& a) { return a; }

// f() is prvalue expr.
int i = f();  // OK
f()   = 7;    // ERROR

// g() is lvalue expr.
int j = g(i); // OK
g(i)  = 3;    // OK => i == 3
\end{minted}
&
\begin{minted}{c++}
int&  k1 = f(); // ERROR
int&& k2 = f(); // OK
k2 += 1;        // OK, rvalue-Reference extends lifetime

int&  l1 = g(i);      // OK
int&& l2 = g(i);      // ERROR, cannot bind lvalue to
                      // rvalue-References.

// const-References
int const& k3 = f();  // OK
k3 += 1;              // ERROR, const-ref. is non-mutable
int const& l3 = g(i); // OK
\end{minted}
\end{tabular}

In the example, the function \cpp{f()} only returns a temporary object (a number), that is not yet assigned to a variable. Rvalue-references can now
be used to refer to this value with a variable before it is finally destroyed.

\begin{example}
Example from \href{https://stackoverflow.com/questions/3716277/do-rvalue-references-allow-dangling-references}{Stackoverflow}:
\begin{minted}{c++}
T&  lvalue();
T   prvalue();
T&& xvalue();

T&& does_not_compile = lvalue();
T&& well_behaved = prvalue();
T&& problematic = xvalue();
\end{minted}
\end{example}

\begin{rem}
  The ultimate usage of rvalue-references is, that you know that the object refereed to is going out of scope or is deleted afterwards, or is a
  temporary at all, thus one can directly take the ownership of that object, without copying its content. This is called a move operation.
\end{rem}


% -------------------------------------------------------------------------------------------------
\subsection{Pointers}
Referencing to something in memory could be understood as just storing the address of that memory. An address ist just an integer indicating
the position in memory relative to some initial address. But, we need more than just the adress, we need the type of the data that we are
referring to, so that we can give the memory a meaning. This is called a pointer:
%
\cppline{TYPE * pointer = &OBJECT;}
%
where \texttt{OBJECT} is any \underline{lvalue-reference} type.
We have used the \texttt{\&} symbol before, to qualify/indicate a reference. In the context of pointers, this symbol is an (unary) operator, the
address-of operator, returning a pointer to the object in memory. A pointer type is indicated by the \texttt{*} qualifier.

The \textit{dual} operator to the address-of operator is the de-reference operator, giving a reference to the object, a pointer points to:
%
\cppline{TYPE & reference = *POINTER;}
%
where \texttt{POINTER} must be any object of pointer type. So, the dereferenced pointer is again an lvalue.

Pointers can be combined with \cpp{const} qualifiers, to either indicate that the addressed data is immutable or that the address value is constant.
Which one you mean is determined by the position of the \cpp{const} qualifier, \ie
%
\begin{minted}{c++}
  int data = 42, data2 = 1234;

  int const* i1 = &data;
  *i1 = 7;                // ERROR: can not modify constant data
  i1 = &data2;            // OK: i1 now points to data2

  int* const i2 = &data;
  *i2 = 7;                // OK: data is not const
  i2 = &data2;            // ERROR: can not change address of const pointer

  int const* const i3 = &data; // completely imutable
\end{minted}

Since one can not have a reference of a reference, one can also not declare a pointer to a reference (something like \cpp{int&*}), but a
pointer is a regular type representing an address, it can be referenced, \ie
%
\begin{minted}{c++}
  int data = 42, data2 = 1234;
  int* p = &data;
  int*& r = p;  // reference to pointer p

  *r = 7;       // changes value of data;
  r = &data2;   // changes stored address of p
\end{minted}

\begin{guideline}{Guideline}
  Don't use pointers to pass data to functions, only if you have a strong reason to do so. Some style-guides use pointers as output arguments
  in functions to not mix it up with input arguments. This is, because you have to explicitly de-reference the pointer in order to use it.
\end{guideline}
